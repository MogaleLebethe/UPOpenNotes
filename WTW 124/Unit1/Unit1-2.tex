\subsection{A Mathematical Model for Space}


In this section, we explore how the mathematical structure known as $\mathbb{R}^3$ serves as a model for 
representing three-dimensional space. Our primary goal is to explain how vectors in $\mathbb{R}^3$ 
(that is, ordered triples of real numbers) can be used to describe the location, or \textit{position}, 
of points within a spatial environment. To illustrate this idea, consider once again a familiar physical setting: 
a rectangular room.

\vspace{1em}

Now to determine the position of a point within this room, we typically measure its perpendicular distances from three fixed 
reference surfaces: two adjacent, non-opposing walls, and the floor. These measurements, denoted $x_1$, $x_2$, and $x_3$, 
effectively capture how far the point lies from each of these surfaces, and they can be expressed as a 
vector $\bar{x} = \langle x_1, x_2, x_3 \rangle$, which resides in $\mathbb{R}^3$.

\vspace{1em}

However, it is important to recognise that this mathematical framework is a simplification, a kind of idealisation, of physical space. 
While it allows for clear and consistent modelling of position, it abstracts away the complexities and imperfections of the real 
world. For instance, the notion of assigning a precise position to a large object, such as a bed, within a small room, is problematic 
under this model, because the object occupies a volume rather than a single point. Conversely, it is more appropriate to model the position
of a small object like a mosquito using a point in $\mathbb{R}^3$, even though, in reality, the mosquito itself is not dimensionless. 
Thus, while the mathematical use of vectors in $\mathbb{R}^3$ offers a powerful tool for analysing space, it operates only within the 
boundaries of idealised assumptions.

\subsubsection{The Model of Space}


So let us now make precise our mathematical model for space. We assume that we know what
 a \textit{point in space} is, what is meant by the \textit{distance between two points}, by \textit{direction} and
 \textit{perpendicular}, and that the \textit{Theorem of Pythagoras} holds.

\begin{definitionbox}
\textbf{The Model of Space}

Fix a point of reference in space, and three mutually perpendicular
directions, labeled $x, y$ and $z$, respectively.
\begin{enumerate}
  \item Define the reference point in space as the zero vector $\bar{0}$ in $\mathbb{R}^3$.
  
  \item An algebraic vector $\bar{x} = \langle x_1, x_2, x_3 \rangle$ in $\mathbb{R}^3$ represents the point in space
        reached by starting at $\bar{0}$ and then:
        \begin{itemize}
          \item moving a distance of $a$ units in the $x$-direction if $a \geq 0,$ or $|a|$ units in the opposite direction if $a < 0$;
          \item followed by a movement of $b$ units in the $y$-direction if $b \geq 0,$ or $|b|$ units in the opposite direction if $b < 0$;
          \item and finally, moving $c$ units in the $z$-direction if $c \geq 0,$ or $|c|$ units in the opposite direction if $c < 0$.
        \end{itemize}
\end{enumerate}
\end{definitionbox}

\tdplotsetmaincoords{65}{120}

\begin{center}
\begin{tikzpicture}[
    scale=5, % Increased size
    tdplot_main_coords,
    axis/.style={->,blue,thick},
    vector/.style={-stealth,red,very thick},
    vector guide/.style={dashed,red,thick}
]

% Define origin
\coordinate (O) at (0,0,0);

% Define vector endpoint
\pgfmathsetmacro{\ax}{1}
\pgfmathsetmacro{\ay}{1}
\pgfmathsetmacro{\az}{1}
\coordinate (P) at (\ax,\ay,\az);

% Axes
\draw[axis] (0,0,0) -- (1,0,0) node[anchor=north east]{$x$};
\draw[axis] (0,0,0) -- (0,1,0) node[anchor=north west]{$y$};
\draw[axis] (0,0,0) -- (0,0,1) node[anchor=south]{$z$};

% Vector and guides
\draw[vector] (O) -- (P);
\draw[vector guide]         (O) -- (\ax,\ay,0);
\draw[vector guide] (\ax,\ay,0) -- (P);
\draw[vector guide]         (P) -- (0,0,\az);
\draw[vector guide] (\ax,\ay,0) -- (0,\ay,0);
\draw[vector guide] (\ax,\ay,0) -- (\ax,0,0);

% Labels for projections
\node[tdplot_main_coords,anchor=east]  at (\ax,0,0) {(\ax, 0, 0)};
\node[tdplot_main_coords,anchor=west]  at (0,\ay,0) {(0, \ay, 0)};
\node[tdplot_main_coords,anchor=south] at (0,0,\az) {(0, 0, \az)};

\end{tikzpicture}
\end{center}

\begin{notebox}
\begin{enumerate}
    \item Since we identify algebraic vectors in $\mathbb{R}^3$ with points in three-dimensional space, we may refer to any algebraic vector $\bar{p} \in \mathbb{R}^3$ as a \textit{point in space}, or simply a \textit{point in} $\mathbb{R}^3$.

    \item It follows from the Theorem of Pythagoras that the distance between the origin $\bar{0}$ and a point $\bar{p}$ is given by $\|\bar{p}\|$.

    \item In general, the distance between two points $\bar{p}$ and $\bar{q}$ is $\|\bar{p} - \bar{q}\|$. Note that $\|\bar{p} - \bar{q}\| = \|\bar{q} - \bar{p}\|$.

    \item When we use an algebraic vector to represent a point in space, we denote its components by lowercase Roman characters. For instance, we may write $\bar{p} = \langle a, b, c \rangle$ or $\bar{x} = \langle x, y, z \rangle$.

    \vspace{0.5em}
    For a point $\bar{p} = \langle a, b, c \rangle$ in space, we call the components of the vector $\bar{p}$ the \textit{Cartesian coordinates} of the point. Specifically, $a$ is the $x$-coordinate of $\bar{p}$, $b$ is the $y$-coordinate, and $c$ is the $z$-coordinate.
\end{enumerate}
\end{notebox}

\begin{examplebox}
  The distance between the points $\bar{p} = \langle 3, 2, -1 \rangle$ and $\bar{q} = \langle 1, 4, 0 \rangle$ is
\[
\|\bar{p} - \bar{q}\| = \|\langle 2, -2, -1 \rangle\| = \sqrt{2^2 + (-2)^2 + (-1)^2} = \sqrt{4 + 4 + 1} = \sqrt{9} = 3.
\]
\end{examplebox}

Our everyday experience tells us that, given points $\bar{p}$, $\bar{q}$, and $\bar{r}$, the distance between $\bar{p}$ and $\bar{q}$ is strictly less than the sum of the distances between $\bar{p}$ and $\bar{r}$ and between $\bar{r}$ and $\bar{q}$, unless $\bar{r}$ lies \textit{between} $\bar{p}$ and $\bar{q}$. 

If our model for space is to be meaningful and useful, it should be consistent with this intuitive observation. The following theorem partially addresses this issue.

\begin{theorembox}
\textbf{The Triangle Inequality for Points in Space}

If $\bar{p}, \bar{q}, \bar{r} \in \mathbb{R}^3$, then
\[
\|\bar{p} - \bar{q}\| \leq \|\bar{p} - \bar{r}\| + \|\bar{r} - \bar{q}\|.
\]
\end{theorembox}

\begin{proofbox}
Assume $\bar{p}, \bar{q}, \bar{r} \in \mathbb{R}^3$.

We want to show that $\|\bar{p} - \bar{q}\| \leq \|\bar{p} - \bar{r}\| + \|\bar{r} - \bar{q}\|$.

\quad $||\bar{p} - \bar{q}|| = ||\bar{p} - \bar{r} + \bar{r} - \bar{q}||$ \hfill [Rearranging terms]

\quad $\Rightarrow ||(\bar{p} - \bar{r}) + (\bar{r} - \bar{q})|| \leq \|\bar{p} - \bar{r}\| + \|\bar{r} - \bar{q}\|$ \hfill [By the Triangle Inequality for the Norm]

\quad $\therefore ||\bar{p} - \bar{q}|| \leq \|\bar{p} - \bar{r}\| + \|\bar{r} - \bar{q}\|$.

\hfill $\qed$

\end{proofbox}

\begin{examplebox}
Let $\bar{p} = \langle 2, 0, 2 \rangle$, \quad
$\bar{q} = \langle 0, 1, 0 \rangle$, \quad
and $\bar{r} = \langle -2, 0, -2 \rangle$.

\vspace{1em}

Then $\|\bar{p} - \bar{q}\| = \|\langle 2, -1, 2 \rangle\| = \sqrt{4 + 1 + 4} = \sqrt{9} = 3$,

\quad $\|\bar{p}\| = \sqrt{2^2 + 0^2 + 2^2} = \sqrt{8} = 2\sqrt{2}$,

\quad and $\|\bar{q}\| = \sqrt{0^2 + 1^2 + 0^2} = \sqrt{1} = 1$.

\vspace{1em}


Hence, $\|\bar{p} - \bar{q}\| = 3 \quad < \quad 2\sqrt{2} + 1 = \|\bar{p}\| + \|\bar{q}\|$.

\vspace{1em}

On the other hand, $\|\bar{p} - \bar{r}\| = \|\langle 4, 0, 4 \rangle\| = \sqrt{16 + 0 + 16} = \sqrt{32} = 4\sqrt{2}$,  

\quad and $\|\bar{p}\| = \|\bar{r}\| = \sqrt{4 + 0 + 4} = \sqrt{8} = 2\sqrt{2}$.

\vspace{1em}

Therefore, $\|\bar{p} - \bar{r}\| = \|\bar{p}\| + \|\bar{r}\|$,  
\quad since $4\sqrt{2} = 2\sqrt{2} + 2\sqrt{2}$.

\end{examplebox}

%ExercisesUnit1-2.tex
% This file contains exercises for Unit 1-2 of WTW124.

\begin{exercisebox}
\begin{enumerate}[label=\arabic*., series=exercises]

\item In each case, calculate the distance between the points $\bar{p}$ and $\bar{q}$
      Determine whether the distance between $\bar{p}$ and $\bar{q}$ is less than the 
      distance between $\bar{p}$ and $\bar{0}$ plus the distance between $\bar{0}$ and $\bar{q}$.

      \begin{enumerate}[label=(\alph*)]
        \item $\bar{p} = \langle 1, 2, 2 \rangle,$ \quad $\bar{q} = \langle 2, 0, -1 \rangle$
        \item $\bar{p} = \langle 2, -1, 2 \rangle,$ \quad $\bar{q} = \langle -4, 2, -4 \rangle$
        \item $\bar{p} = \langle 3, 1, -1 \rangle,$ \quad $\bar{q}\langle 1, 2, 3 \rangle$
      \end{enumerate}

\item Consider the points $\bar{p} = \langle 1, \alpha, 2 \rangle,$ \quad $\bar{q} = \langle 1, 0, 4\rangle$ where $\alpha \in \mathbb{R}$.
      Find the value of $\alpha$ if:
      \begin{enumerate}[label=(\alph*)]
        \item the distance between $\bar{p}$ and $\bar{q}$ is 3 units.
        \item the distance between $\bar{p}$ and $\bar{q}$ is 1 unit.
      \end{enumerate}

\item Let $\bar{p} = \langle 1, 2, 1 \rangle,$ \quad $\bar{q} = \langle -1, 0, -1 \rangle$ and $\bar{r}\langle x, y, z \rangle$.\\
      Show that $||\bar{p} - \bar{r}|| = ||\bar{q} - \bar{r}||$ if and only if $x+y+z=1$

\end{enumerate}
\end{exercisebox}

In the next unit, we will use our model of space to define lines.

\newpage

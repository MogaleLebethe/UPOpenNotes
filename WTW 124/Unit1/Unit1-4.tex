\subsection{Angles and Measurement}

Angles are common in the world around us: the side line and the goal line on a soccer pitch
form an angle at the point where they intersect, as do non-opposing walls in a rectangular
room. In this section we define angles in the context of our model for space, and introduce
angle measure. As a motivation for our definitions of angle and the magnitude of an angle,
we introduce triangles, and show that our definitions are consistent with the Cosine Law.

\begin{definitionbox}
\textbf{Angle}

Let \(R_1\) and \(R_2\) be two rays \(R_1 = \{\bar{a} + t\bar{p} : t \geq 0\}\) and \(R_2 = \{\bar{a} + t\bar{q} : t \geq 0\}\) with the same
origin \(\bar{a}\). The set \(R_1 \cup R_2\) of all points of all points on the two rays is called the \textbf{angle} formed by the rays \(R_1\) and \(R_2\).
\end{definitionbox}

The following figure shows the angle formed by the rays \(R_1\) and \(R_2\).

\begin{center}
    \begin{tikzpicture}
        \coordinate (R1) at (3, 3);
        \coordinate (A) at (0, 0);
        \coordinate (R2) at (4.5, 0);
        \coordinate (MdptR1) at (1.5, 1.5);
        \coordinate (MdptR2) at (2.25, 0);
 
        
        \draw[-{Stealth[length=3mm, width=2mm]}, thick] (A) -- (R1) node[above left] {$R_1$};
        \draw[-{Stealth[length=3mm, width=2mm]}, thick] (A) -- (R2) node[above right] {$R_2$};
        \fill (A) circle (2pt) node[below left] {$\bar{a}$};
        \fill (MdptR1) circle (2pt) node[above left] {$\bar{a}+\bar{p}$};
        \fill (MdptR2) circle (2pt) node[below right] {$\bar{a}+\bar{q}$};
 

    \end{tikzpicture}
\end{center}

\begin{remarkbox}
    Take note of the following regarding rays.

    \begin{enumerate}
        \item Consider two rays \(R_1 = \{\bar{a} + t\bar{p} : t \geq 0\}\) and \(R_2 = \{\bar{a} + t\bar{q} : t \geq 0\}\) with the same
origin \(\bar{a}\). If the direction vectors \(\bar{p}\) and \(\bar{q}\) are scalar multiples of each other, then \(R_1 = R_2\), or \(R_1 \cup R_2\) is a line.

        \item Let \(\bar{x}, \bar{y}, \bar{z}\) be three points not on the same line. Then the line segments \\
            \(S_1 = \{\bar{x}+t(\bar{y}-\bar{x}) : 0 \leq t \leq 1\}\) and \(S_2 = \{\bar{x}+t(\bar{z}-\bar{x}) : 0 \leq t \leq 1\}\) determine the angle formed by the rays 
            \(R_1 =\{\bar{x}+t(\bar{y}-\bar{x}): t \geq 0\}\) and \(R_2 =\{\bar{x}+t(\bar{z}-\bar{x}): t \geq 0\}\):
        \begin{center}
    \begin{tikzpicture}
        \coordinate (R1) at (3, 3);
        \coordinate (A) at (0, 0);
        \coordinate (R2) at (4.5, 0);
        \coordinate (MdptR1) at (1.5, 1.5);
        \coordinate (MdptR2) at (2.25, 0);
 
        
        \draw[-{Stealth[length=3mm, width=2mm]}, thin] (A) -- (R1) node[above left] {$R_1$};
        \draw[-{Stealth[length=3mm, width=2mm]}, thin] (A) -- (R2) node[above right] {$R_2$};
        \draw[-, very thick] (A) -- (MdptR1);
        \draw[-, very thick] (A) -- (MdptR2);
        
        \fill (A) circle (2pt) node[below left] {$\bar{x}$};
        \fill (MdptR1) circle (2pt) node[above left] {$\bar{y}$};
        \fill (MdptR2) circle (2pt) node[below right] {$\bar{z}$};
 

    \end{tikzpicture}
\end{center}  
        
    \end{enumerate}
\end{remarkbox}

\newpage

Our aim now is to define the magnitude of an angle; that is, to measure the size of an angle, 
because, in the figure below, the angle on the right is clearly `larger' than the one on the left.

\begin{center}
    
    \begin{tikzpicture}
    \coordinate (R1) at (1.5, 3);
    \coordinate (A) at (0, 0);
    \coordinate (R2) at (4.5, 0); 

    \draw[-{Stealth[length=3mm, width=2mm]}, thick] (A) -- (R1) node[above left] {$R_1$};
    \draw[-{Stealth[length=3mm, width=2mm]}, thick] (A) -- (R2) node[above right] {$R_2$};
\end{tikzpicture}
\begin{tikzpicture}
    \coordinate (R1) at (4, 3);
    \coordinate (A) at (0, 0);
    \coordinate (R2) at (4.5, 0);

    \draw[-{Stealth[length=3mm, width=2mm]}, thick] (A) -- (R1) node[above left] {$R_1$};
    \draw[-{Stealth[length=3mm, width=2mm]}, thick] (A) -- (R2) node[above right] {$R_2$};

\end{tikzpicture}

\end{center}

Now we need to quantify the what we refer to as the `size of an angle'.

\vspace{1em}

First note that for vectors \(\bar{x}, \bar{y}\), such that \(\bar{x},\bar{y} \neq 0\), by the Cauchy-Schwarz Inequality,
\(|\bar{x}\cdot \bar{y}| \leq \|\bar{x}\| \|\bar{y}\| \).

Then, by the Positive Definiteness of the Norm and the definition of the absolute value, we have

\[
    \bigg|\frac{\bar{x}\cdot \bar{y}}{\|\bar{x}\| \|\bar{y}\|}\bigg| \leq 1
\]

Finally, we find that:

\[
-1 \leq \frac{\bar{x}\cdot \bar{y}}{\|\bar{x}\| \|\bar{y}\|}\leq 1
\]

Now consider the function \(f : [0,\pi] \rightarrow \mathbb{R}\) defined by \(f(\theta)= \cos \theta\). Then \(f\) is continuous and strictly decreasing 
on the interval \([0,\pi]\), with \(f(0) = 1\) and \(f(\pi) = -1\). By the Intermediate Value Theorem, and the fact that $f$ is strictly decreasing, there exists exactly one \( \theta \in [0,\pi] \) that satisfies 

\[
    f(\theta) = \cos \theta = \frac{\bar{x}\cdot \bar{y}}{\|\bar{x}\| \|\bar{y}\|}
\]

This leads to our definition of the magnitude of an angle:

\begin{definitionbox}
\textbf{Magnitude of an Angle}

Consider the angle formed by the rays with equations \(\bar{x} = \bar{a} + t\bar{p}\) and \(\bar{x} = \bar{a} + t\bar{q}\), with \(t \geq 0\).
Then the \textit{magnitude of the angle} is the unique real number \(\theta \in [0,\pi]\) such that

\[
    \cos \theta = \frac{\bar{p}\cdot \bar{q}}{\|\bar{p}\| \|\bar{q}\|}
\]
\end{definitionbox}

\begin{remarkbox}
    Consider the angle formed by the rays \(R_1\) and \(R_2\) with equations \(\bar{x} = \bar{a} + t\bar{p}\) and \(\bar{x} = \bar{a} + t\bar{q}\), with \(t \geq 0\), respectively.
    Let \(\theta\) be the magnitude of the angle. Then

    \[
        |\bar{p}\cdot\bar{q}| = \| \bar{p}\|\cdot \| \bar{q}\| \text{ if and only if } \bar{p} \text{ and } \bar{q} \text{ are scalar multiples of each other.}
    \]

    In particular, 
    \(
        \bar{p}\cdot\bar{q} = \| \bar{p}\|\| \bar{q}\| \text{ if and only if } \bar{p} = \alpha\bar{q} \text{ for some } \alpha > 0,
    \)

    and 
    \(
        \bar{p}\cdot\bar{q} = -\| \bar{p}\|\| \bar{q}\| \text{ if and only if } \bar{p} = \alpha\bar{q} \text{ for some } \alpha < 0.
    \)

    We therefore have the following:
    \begin{enumerate}
        \item If \(\bar{p}\) and \(\bar{q}\) are not scalar multiples of each other, then \(\theta \in (0,\pi)\).
        \item If \(\bar{p} = \alpha\bar{q}\) for some \(\alpha > 0\), then \(\theta = 0\). In this case, \(R_1=R_2\), so the angle \(R_1 \cup R_2\) is a ray.
        \item If \(\bar{p} = \alpha\bar{q}\) for some \(\alpha < 0\), then \(\theta = \pi\). In this case, \(R_1 \cup R_2\) is a line.
    \end{enumerate}

\end{remarkbox}

\begin{examplebox}
Consider the rays \( R_1 \) and \( R_2 \) with equations
\(
\bar{x} = \bar{0} + t\langle 1, -1, 1 \rangle \text{ and } \bar{x} = \bar{0} + t\langle 2\alpha, 1, 1 \rangle, t \geq 0,
\)
respectively. Find all values for \( \alpha \in \mathbb{R} \) so that the angle formed by \( R_1 \) and \( R_2 \) has magnitude \( \frac{\pi}{3} \).

\vspace{0.5em}
\textbf{Solution}

\vspace{0.5em}
The direction vector for \( R_1 \) is \( \bar{p} = \langle 1, -1, 1 \rangle \), while that for \( R_2 \) is \( \bar{q} = \langle 2\alpha, 1, 1 \rangle \).

The magnitude of the angle formed by these two rays is \( \frac{\pi}{3} \) if and only if


\[
\cos\left(\frac{\pi}{3}\right) = \frac{\bar{p} \cdot \bar{q}}{\|\bar{p}\| \|\bar{q}\|} = \frac{2\alpha - 1 + 1}{\sqrt{3} \sqrt{4\alpha^2 + 2}} = \frac{2\alpha}{\sqrt{3} \sqrt{4\alpha^2 + 2}}.
\]


Since \( \cos\left(\frac{\pi}{3}\right) = \frac{1}{2} \), we equate:


\[
\frac{1}{2} = \frac{2\alpha}{\sqrt{3} \sqrt{4\alpha^2 + 2}}.
\]


Multiplying both sides by \( 2\sqrt{3} \sqrt{4\alpha^2 + 2} \), we get:


\[
\sqrt{3} \sqrt{4\alpha^2 + 2} = 4\alpha.
\]


Squaring both sides:


\[
3(4\alpha^2 + 2) = 16\alpha^2 \quad \Rightarrow \quad 12\alpha^2 + 6 = 16\alpha^2 \quad \Rightarrow \quad 4\alpha^2 = 6 \quad \Rightarrow \quad \alpha^2 = \frac{3}{2}.
\]


Thus, \( \alpha = \pm \frac{\sqrt{6}}{2} \). However, we must check which of these satisfy the original equation.

Substituting \( \alpha = \frac{\sqrt{6}}{2} \) into the original equation confirms it satisfies the condition. But \( \alpha = -\frac{\sqrt{6}}{2} \) does not, because the cosine would be negative, implying an angle greater than \( \frac{\pi}{2} \).

Therefore, the only value for \( \alpha \) for which the angle formed by the rays \( R_1 \) and \( R_2 \) is \( \frac{\pi}{3} \) is
\(
\alpha = \frac{\sqrt{6}}{2}.
\)

\end{examplebox}

\newpage

Now that we have defined angles and lines in terms of our model of space, we can define a triangle. 

\begin{definitionbox}
\textbf{Triangle}

    Let \(\bar{a}, \bar{b}, \bar{c} \in \mathbb{R}^3,\) not all on the same line. Let \(S_1\) be 
    the line segment between \(\bar{a}\) and \(\bar{b}\), \(S_2\) be the line segment between
    \(\bar{b}\) and \(\bar{c}\), and \(S_3\) be the line segment between \(\bar{a}\) and \(\bar{c}\).
    Then the \textit{triangle} with vertices \(\bar{a}\), \( \bar{b}\), and \( \bar{c}\) is the set \(S_1 \cup S_2 \cup S_3\) 
    of points on the three line segments.
\end{definitionbox}

The figure below illustrates a triangle, as defined above:

 
\begin{center}
    \begin{tikzpicture}[scale=0.8]
        \coordinate (A) at (-2,0);
        \coordinate (B) at (6,0);
        \coordinate (C) at (2,4);
        \coordinate (S1) at (0,2);
        \coordinate (S2) at (3,2);
        \coordinate (S3) at (0,0);

        
        \draw[very thick] (A) -- (B) node[midway, below] {};
        \draw[very thick] (B) -- (C) node[midway, right] {};
        \draw[very thick] (C) -- (A) node[midway, left] {};
        \fill (A) circle (2pt) node[below left] {$\bar{a}$};
        \fill (B) circle (2pt) node[below right] {$\bar{b}$};
        \fill (C) circle (2pt) node[above left] {$\bar{c}$};

        \node at ($(A)!0.5!(C)$) [left] {$S_1$};
        \node at ($(A)!0.5!(B)$) [below] {$S_3$};
        \node at ($(B)!0.5!(C)$) [right] {$S_2$};
        
    \end{tikzpicture}
\end{center}

\begin{remarkbox}
    Consider three points \(\bar{a}, \bar{b}, \bar{c} \in \mathbb{R}^3\), not on the same line, and the
    triangle with vertices \(\bar{a}\), \(\bar{b}\), and \(\bar{c}\).

    \begin{enumerate}
        \item The line segments \(S_1, S_2,\) and \(S_3\), joining \(\bar{a}, \bar{b},\) and \(\bar{c}\) respectively,
        are called the \textit{sides} of the triangle.
        
        \item The triangle determines three angles:
        \begin{enumerate}[label=(\alph*)]
            \item The angle at \(\bar{a}\), formed by the rays with equations 
            \(\bar{x} = \bar{a} + t(\bar{b} - \bar{a})\) and \(\bar{x} = \bar{a} + t(\bar{c} - \bar{a})\), 
            where \(t \geq 0\).
            
            \item The angle at \(\bar{b}\), formed by the rays with equations 
            \(\bar{x} = \bar{b} + t(\bar{a} - \bar{b})\) and \(\bar{x} = \bar{b} + t(\bar{c} - \bar{b})\), 
            where \(t \geq 0\).
            
            \item The angle at \(\bar{c}\), formed by the rays with equations 
            \(\bar{x} = \bar{c} + t(\bar{a} - \bar{c})\) and \(\bar{x} = \bar{c} + t(\bar{b} - \bar{c})\), 
            where \(t \geq 0\).
            The triangle with vertices \(\bar{a}\), \(\bar{b}\), and \(\bar{c}\) and the three angles
            determined by the triangle are illustrated below:
            \begin{center}
\begin{multicols}{3}
% Angle at a
\begin{tikzpicture}[scale=0.5]
    \coordinate (A) at (0,0);
    \coordinate (B) at (2,3);
    \coordinate (C) at (4,0);

    \draw[thick] (A) -- (B) -- (C) -- cycle;

    \fill (A) circle (2pt) node[below left] {$\bar{a}$};
    \fill (B) circle (2pt) node[above] {$\bar{b}$};
    \fill (C) circle (2pt) node[below right] {$\bar{c}$};

    \draw[->, thick, color=blue] (A) -- ($(A)!0.5!(B)$);
    \draw[->, thick, color=blue] (A) -- ($(A)!0.5!(C)$);

    \node[below=5pt] at (2,-0.5) {\small Angle at $\bar{a}$};
\end{tikzpicture}

% Angle at b
\begin{tikzpicture}[scale=0.5]
    \coordinate (A) at (0,0);
    \coordinate (B) at (2,3);
    \coordinate (C) at (4,0);

    \draw[thick] (A) -- (B) -- (C) -- cycle;

    \fill (A) circle (2pt) node[below left] {$\bar{a}$};
    \fill (B) circle (2pt) node[above] {$\bar{b}$};
    \fill (C) circle (2pt) node[below right] {$\bar{c}$};

    \draw[->, thick, color=red] (B) -- ($(B)!0.5!(A)$);
    \draw[->, thick, color=red] (B) -- ($(B)!0.5!(C)$);

    \node[below=5pt] at (2,-0.5) {\small Angle at $\bar{b}$};
\end{tikzpicture}

% Angle at c
\begin{tikzpicture}[scale=0.5]
    \coordinate (A) at (0,0);
    \coordinate (B) at (2,3);
    \coordinate (C) at (4,0);

    \draw[thick] (A) -- (B) -- (C) -- cycle;

    \fill (A) circle (2pt) node[below left] {$\bar{a}$};
    \fill (B) circle (2pt) node[above] {$\bar{b}$};
    \fill (C) circle (2pt) node[below right] {$\bar{c}$};

    \draw[->, thick, color=green!70!black] (C) -- ($(C)!0.5!(A)$);
    \draw[->, thick, color=green!70!black] (C) -- ($(C)!0.5!(B)$);

    \node[below=5pt] at (2,-0.5) {\small Angle at $\bar{c}$};
\end{tikzpicture}
\end{multicols}
            \end{center}
        \end{enumerate}
    \end{enumerate}
\end{remarkbox}

\newpage

Our definition of the magnitude of an angle is consistent with the Cosine Rule.

\begin{propositionbox}
    \textbf{Cosine Rule}

    Consider a triangle with vertices \(\bar{a}\), \(\bar{b}\), and \(\bar{c}\). Let \(\theta\) be the magnitude of the angle at \(\bar{b}\). 
    \begin{center}
    \begin{tikzpicture}[scale=0.8]
        \coordinate (A) at (-2,0);
        \coordinate (B) at (6,0);
        \coordinate (C) at (2,4);
        \coordinate (S1) at (0,2);
        \coordinate (S2) at (3,2);
        \coordinate (S3) at (0,0);

        
        \draw[very thick] (A) -- (B) node[midway, below] {};
        \draw[very thick] (B) -- (C) node[midway, right] {};
        \draw[very thick] (C) -- (A) node[midway, left] {};
        \fill (A) circle (2pt) node[below left] {$\bar{a}$};
        \fill (B) circle (2pt) node[below right] {$\bar{b}$};
        \fill (C) circle (2pt) node[above left] {$\bar{c}$};

        \node at ($(A)!0.5!(C)$) [left] {$S_1$};
        \node at ($(A)!0.5!(B)$) [below] {$S_3$};
        \node at ($(B)!0.5!(C)$) [right] {$S_2$};
        
    \end{tikzpicture}
\end{center}

    Then we have that
    
    \[
        \|\bar{a}-\bar{c}\|^2 = \|\bar{a}-\bar{b}\|^2 + \|\bar{b}-\bar{c}\|^2 - 2\|\bar{a}-\bar{b}\| \|\bar{b}-\bar{c}\| \cos \theta.
    \]    
\end{propositionbox}

\begin{proofbox}
    Calculating the square of the length of the side \(S_3\) we have:
    \[
\begin{aligned}
\|\bar{a} - \bar{c}\|^2 
&= \|(\bar{a} - \bar{b}) - (\bar{c} - \bar{b})\|^2 \\
&= \left[(\bar{a} - \bar{b}) - (\bar{c} - \bar{b})\right] \cdot \left[(\bar{a} - \bar{b}) - (\bar{c} - \bar{b})\right] \\
&= (\bar{a} - \bar{b}) \cdot (\bar{a} - \bar{b}) 
   - 2\left[(\bar{a} - \bar{b}) \cdot (\bar{c} - \bar{b})\right] 
   + (\bar{c} - \bar{b}) \cdot (\bar{c} - \bar{b}) \\
&= \|\bar{a} - \bar{b}\|^2 
   - 2\left[(\bar{a} - \bar{b}) \cdot (\bar{c} - \bar{b})\right] 
   + \|\bar{c} - \bar{b}\|^2
\end{aligned}
\]

Note that the angle at \( \bar{b} \) is determined by the rays with equations
\[
\bar{x} = \bar{b} + t(\bar{a} - \bar{b}) \quad \text{and} \quad \bar{x} = \bar{b} + t(\bar{c} - \bar{b}), \quad t \geq 0.
\]
Therefore, calculating the magnitude \( \theta \) of the angle at \( \bar{b} \), we find
\[
\cos\theta = \frac{(\bar{a} - \bar{b}) \cdot (\bar{c} - \bar{b})}{\|\bar{a} - \bar{b}\| \, \|\bar{c} - \bar{b}\|}.
\]
Hence,
\[
\|\bar{a} - \bar{c}\|^2 = \|\bar{a} - \bar{b}\|^2 + \|\bar{c} - \bar{b}\|^2 - 2\|\bar{a} - \bar{b}\| \, \|\bar{c} - \bar{b}\| \cos\theta.
\]
\end{proofbox}
\begin{examplebox}
Consider the triangle with vertices $\bar{a} = \langle 1,0,1 \rangle$, $\bar{b} = \langle 1,2,1 \rangle$, and $\bar{c} = \langle 1,2,3 \rangle$. We determine the magnitude of the angle at $\bar{a}$.

\vspace{0.5em}
\textbf{Solution.} Let $\theta$ be the magnitude of the angle at $\bar{a}$. According to the cosine rule,
\[
\|\bar{b} - \bar{c}\|^2 = \|\bar{b} - \bar{a}\|^2 + \|\bar{c} - \bar{a}\|^2 - 2\|\bar{b} - \bar{a}\|\|\bar{c} - \bar{a}\|\cos\theta.
\]
Hence,
\[
4 = 4 + 8 - 8\sqrt{2}\cos\theta \quad \Rightarrow \quad \cos\theta = \frac{1}{\sqrt{2}}.
\]
Therefore, $\theta = \frac{\pi}{4}$.
\end{examplebox}

\begin{examplebox}
Consider the triangle with vertices $\bar{a} = \langle 1,0,1 \rangle$, $\bar{b} = \langle 1,2,1 \rangle$, and $\bar{c} = \langle \alpha,1,0 \rangle$. We find the values for $\alpha$ so that the angle at $\bar{a}$ has magnitude $\frac{\pi}{3}$.

\vspace{0.5em}
\textbf{Solution.} According to the cosine rule,
\[
\|\bar{b} - \bar{c}\|^2 = \|\bar{b} - \bar{a}\|^2 + \|\bar{c} - \bar{a}\|^2 - 2\|\bar{b} - \bar{a}\|\|\bar{c} - \bar{a}\|\cos\left(\frac{\pi}{3}\right).
\]
We have:
\[
\|\bar{b} - \bar{a}\| = 2, \quad \|\bar{c} - \bar{a}\| = \sqrt{(\alpha - 1)^2 + 2}, \quad \|\bar{b} - \bar{c}\| = \sqrt{(\alpha - 1)^2 + 2}.
\]
Substituting:
\[
(\alpha - 1)^2 + 2 = 4 + (\alpha - 1)^2 + 2 - 2 \cdot 2 \cdot \sqrt{(\alpha - 1)^2 + 2} \cdot \frac{1}{2}.
\]
Simplifying:
\[
(\alpha - 1)^2 + 2 = 6 + (\alpha - 1)^2 + 2 - 2\sqrt{(\alpha - 1)^2 + 2},
\]
\[
2\sqrt{(\alpha - 1)^2 + 2} = 6 \quad \Rightarrow \quad \sqrt{(\alpha - 1)^2+2}=3 \quad \Rightarrow \quad (\alpha - 1)^2 = 7.
\]
Hence, $\alpha = 1 \pm \sqrt{7}$.
\end{examplebox}

\begin{examplebox}
Consider the triangle with vertices $\bar{a} = \langle 1,-1,1 \rangle$, $\bar{b} = \langle 2,1,1 \rangle$, and $\bar{c} = \langle -2\alpha - 1, \alpha, 6 \rangle$. We find the values for $\alpha$ so that the angle at $\bar{b}$ has magnitude $\frac{\pi}{4}$.

\vspace{0.5em}
\textbf{Solution.} Using the cosine rule:
\[
\|\bar{a} - \bar{c}\|^2 = \|\bar{a} - \bar{b}\|^2 + \|\bar{c} - \bar{b}\|^2 - 2\|\bar{a} - \bar{b}\|\|\bar{c} - \bar{b}\|\cos\left(\frac{\pi}{4}\right).
\]
We compute:
\[
\|\bar{a} - \bar{b}\| = \sqrt{5}, \quad \|\bar{a} - \bar{c}\| = \sqrt{5\alpha^2 + 10\alpha + 30}, \quad \|\bar{c} - \bar{b}\| = \sqrt{5\alpha^2 + 10\alpha + 35}.
\]
Substituting:
\[
5\alpha^2 + 10\alpha + 30 = 5 + 5\alpha^2 + 10\alpha + 35 - 2\sqrt{5}\sqrt{5\alpha^2+10\alpha+35}\cos\left(\frac{\pi}{4}\right)
\]
\[
5\alpha^2 + 10\alpha + 30 = 5\alpha^2 + 10\alpha + 40 - 2\sqrt{5}\sqrt{5\alpha^2+10\alpha+35}\frac{1}{\sqrt{2}}
\]
\[
-10 = -\sqrt{10}\sqrt{5\alpha^2+10\alpha+35} \quad \Rightarrow \quad 10 = \sqrt{10}\sqrt{5\alpha^2+10\alpha+35}
\]
\[
100 = 10(5\alpha^2+10\alpha+35) \quad \Rightarrow \quad 10=5\alpha^2+10\alpha+35 \quad \Rightarrow \quad 5\alpha^2+10\alpha+25 = 0
\]
\[
\alpha^2+2\alpha+5=0.
\]
Since the discriminant is negative, there are no real solutions. Therefore, no value of $\alpha$ gives an angle of $\frac{\pi}{4}$ at $\bar{b}$.
\end{examplebox}

\begin{examplebox}
Consider a triangle with vertices $\bar{a}$, $\bar{b}$, and $\bar{c}$ such that $\|\bar{a} - \bar{b}\| = 3$, $\|\bar{c} - \bar{a}\| = 4$, and the angle at $\bar{b}$ has magnitude $\frac{\pi}{6}$. We determine $\|\bar{b} - \bar{c}\|$.

\vspace{0.5em}
\textbf{Solution.} By the cosine rule:
\[
\|\bar{a} - \bar{c}\|^2 = \|\bar{c} - \bar{b}\|^2 + \|\bar{a} - \bar{b}\|^2 - 2\|\bar{a} - \bar{b}\|\|\bar{c} - \bar{b}\|\cos\left(\frac{\pi}{6}\right).
\]
Substituting:
\[
16 = 9 + \|\bar{b} - \bar{c}\|^2 - 2(3)\|\bar{b} - \bar{c}\|\left(\frac{\sqrt{3}}{2}\right).
\]
\[
16 = 9 + \|\bar{b} - \bar{c}\|^2 - 3\sqrt{3}\|\bar{b} - \bar{c}\|.
\]
Let $x = \|\bar{b} - \bar{c}\|$. Then:
\[
x^2 - 3\sqrt{3}x - 7 = 0.
\]
Solving using the quadratic formula:
\[
x = \frac{3\sqrt{3} \pm \sqrt{(3\sqrt{3})^2 - 4(1)(-7)}}{2} = \frac{3\sqrt{3} \pm \sqrt{27+28}}{2} = \frac{3\sqrt{3} \pm \sqrt{55}}{2}.
\]
Since $x$ must be a positive length, we take the positive root. Therefore:
\[
\|\bar{b} - \bar{c}\| = \frac{3\sqrt{3} + \sqrt{55}}{2}.
\]
\end{examplebox}
\begin{exercisebox}
\begin{enumerate}[series=AngleEX]
  \item Find the magnitude of the given angle, if it is defined. Otherwise, explain why it is not defined.
  \begin{enumerate}
    \item The angle determined by the rays with equations $\bar{x} = \bar{0} + t\langle 0, 2, 1 \rangle$ and $\bar{x} = \bar{0} + t\langle 3, -1, 2 \rangle$, $t \geq 0$
    \item The angle determined by the rays with equations $\bar{x} = \langle 1, 0, 1 \rangle + t\langle 0, 2, 0 \rangle$ and $\bar{x} = \langle 1, 0, 1 \rangle + t\langle -1, 3, \sqrt{2} \rangle$, $t \geq 0$
    \item The angle determined by the rays with equations $\bar{x} = \langle 1, 2, 3 \rangle + t\langle \sqrt{2}, -3\sqrt{2}, -2 \rangle$ and $\bar{x} = \langle 1, 2, 3 \rangle + t\langle 0, 2, 0 \rangle$, $t \geq 0$
    \item The angle determined by the rays with equations $\bar{x} = \langle -2, 2, 1 \rangle + t\langle 4, 8, 4 \rangle$ and $\bar{x} = \langle -2, 2, 1 \rangle + t\langle 1, 2, 1 \rangle$, $t \geq 0$
    \item The angle determined by the rays with equations $\bar{x} = \langle 3, 2, 1 \rangle + t\langle 1, 0, \frac{1}{\sqrt{3}} \rangle$ and $\bar{x} = \langle 2, 2, 5 \rangle + t\langle 0, 0, -1 \rangle$, $t \geq 0$
    \item The angle determined by the rays with equations $\bar{x} = \langle 2, 2, 5 \rangle + t\langle 3, 0, \sqrt{3} \rangle$ and $\bar{x} = \langle 2, 2, 5 \rangle + t\langle 0, 0, 2 \rangle$, $t \geq 0$
    \item The angle determined by the rays with equations $\bar{x} = \langle 0, -1, -3 \rangle + t\langle 1+\sqrt{3}, 2, 1 - \sqrt{\sqrt{3}} \rangle$ and $\bar{x} = \langle 0, -1, -3 \rangle + t\langle 1, 2, 1 \rangle$, $t \geq 0$
    \item The angle determined by the rays with equations $\bar{x} = \langle 0, -1, -3 \rangle + t\langle -1, 1 + 3, 2 + \sqrt{3} \rangle$ and $\bar{x} = \langle 0, -1, -3 \rangle + t\langle -2, -4, -2 \rangle$, $t \geq 0$
    \item The angle determined by the rays with equations $\bar{x} = \langle 1, 2, 1 \rangle + t\langle 1, 1, 2 \rangle$ and $\bar{x} = \langle 1, 2, 1 \rangle + t\langle -1, -1, 2 \rangle$, $t \geq 0$
  \end{enumerate}

  \item Consider the triangle with vertices $\bar{a}, \bar{b}, \bar{c}$. In each case, find the magnitude of the specified angle.
  \begin{enumerate}
    \item $\bar{a} = \langle 1,1,1 \rangle$, $\bar{b} = \langle 1,3,2 \rangle$, $\bar{c}= \langle 7,-1,5 \rangle$; the angle at $\bar{a}$
    \item $\bar{a} = \langle 1,-1,1 \rangle$, $\bar{b} = \langle 1,3,1 \rangle$, $\bar{c}= \langle 1 - \sqrt{2}, 3\sqrt{2} - 1, 3 \rangle$; the angle at $\bar{a}$
    \item $\bar{a} = \langle 1,1,1 \rangle$, $\bar{b} = \langle 0,1,0 \rangle$, $\bar{c}= \langle 0,0,1 \rangle$; the angle at $\bar{c}$
    \item $\bar{a} = \bar{0}$, $\bar{b} = \langle 2, -6, -2\sqrt{2} \rangle$, $\bar{c}= \langle 0, 1, 0 \rangle$; the angle at $\bar{a}$
    \item $\bar{a} = \langle \sqrt{3}, 3, -\sqrt{3} \rangle$, $\bar{b} = \langle 1, 5, 1 \rangle$, $\bar{c} = \langle -1, 1, -1 \rangle$; the angle at $\bar{c}$
    \item $\bar{a} = \langle 0, 2, -3 \rangle$, $\bar{b} = \langle \sqrt{3}, 2, -2 \rangle$, $\bar{c}= \langle 0, 2, -5 \rangle$; the angle at $\bar{a}$
    \item $\bar{a} = \langle 2, 1, 3\sqrt{3} \rangle$, $\bar{b} = \langle -1, 1, 2\sqrt{3} \rangle$, $\bar{c}= \langle -1, 1, 4\sqrt{3} \rangle$; the angle at $\bar{b}$
  \end{enumerate}

  \item Consider the triangle with vertices $\bar{a} = \bar{0}$, $\bar{b} = \langle 1, 2, 1 \rangle$, and $\bar{c} = \langle 0, \alpha, 1 \rangle$. Find the values of $\alpha$ such that the magnitude of the angle at $\bar{a}$ is:
  \begin{multicols}{2}
  \begin{enumerate}
    \item $\frac{\pi}{6}$
    \item $\frac{\pi}{4}$
    \item $\frac{\pi}{3}$
    \item $\arccos\left( \frac{2\sqrt{2}}{\sqrt{3}} \right)$
    \item $\frac{5\pi}{6}$
    \item $\frac{3\pi}{4}$
    \item $\frac{\pi}{2}$
  \end{enumerate}
  \end{multicols}
  \end{enumerate}
\end{exercisebox}

\begin{exercisebox}
  \begin{enumerate}[resume=AngleEX]  

  \item Consider a triangle with vertices $\bar{a}, \bar{b}, \bar{c}$. Determine the following:
  \begin{enumerate}
    \item $\|\bar{a} - \bar{c}\|$ if $\|\bar{a} - \bar{b}\| = 2$, $\|\bar{c} - \bar{b}\| = 3$, and the angle at $\bar{b}$ has magnitude $\frac{\pi}{6}$.
    \item $\|\bar{a} - \bar{b}\|$ if $\|\bar{a} - \bar{c}\| = 2$, $\|\bar{c} - \bar{b}\| = 3$, and the angle at $\bar{b}$ has magnitude $\frac{\pi}{3}$.
    \item $\|\bar{a} - \bar{b}\|$ if $\|\bar{a} - \bar{c}\| = 3$, $\|\bar{c} - \bar{b}\| = 2$, and the angle at $\bar{b}$ has magnitude $\frac{\pi}{2}$.
    \item $\|\bar{a} - \bar{b}\|$ if $\|\bar{a} - \bar{c}\| = 3$, the angle at $\bar{a}$ is $\frac{\pi}{3}$ and the angle at $\bar{c}$ is $\frac{\pi}{2}$.
  \end{enumerate}

  \item Let $\bar{a}, \bar{b}, \bar{c}$ be the vertices of a triangle. Show that if the angle at $\bar{a}$ has magnitude $\frac{\pi}{2}$, then $\|\bar{b} - \bar{c}\|^2 = \|\bar{a} - \bar{c}\|^2 + \|\bar{a} - \bar{b}\|^2$.

  \item Let $\bar{a}, \bar{b}, \bar{c}$ be the vertices of a triangle. Show that if $\|\bar{b} - \bar{c}\|^2 = \|\bar{a} - \bar{c}\|^2 + \|\bar{a} - \bar{b}\|^2$, then the angle at $\bar{a}$ has magnitude $\frac{\pi}{2}$.

  \item Let $\bar{a}, \bar{b}, \bar{c}$ be the vertices of a triangle. If the angles at $\bar{a}$ and $\bar{c}$ have the same magnitude, show that $\|\bar{b} - \bar{a}\| = \|\bar{b} - \bar{c}\|$.

  \item Let $\bar{a}, \bar{b}, \bar{c}$ be the vertices of a triangle. If $\|\bar{b} - \bar{a}\| = \|\bar{b} - \bar{c}\|$, show that the angles at $\bar{a}$ and $\bar{c}$ have the same magnitude.

  \item Let $\bar{a}, \bar{b}, \bar{c}$ be the vertices of a triangle. If $\|\bar{b} - \bar{a}\| = \|\bar{b} - \bar{c}\| = \|\bar{a} - \bar{c}\|$, show that the angles at $\bar{a}$, $\bar{b}$ and $\bar{c}$ all have the same magnitude.

  \item Let $\bar{a}, \bar{b}, \bar{c}$ be the vertices of a triangle. If the angles at $\bar{a}, \bar{b}, \bar{c}$ all have the same magnitude, show that $\|\bar{b} - \bar{a}\| = \|\bar{b} - \bar{c}\| = \|\bar{a} - \bar{c}\|$.
\end{enumerate}
\end{exercisebox}


\newpage
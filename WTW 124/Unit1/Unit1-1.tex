\subsection{Algebraic Vectors and Vector Algebra}

\subsubsection{Algebraic Vectors}


In this section, we formalise the ideas presented in the introduction.
At this point our definitions and theorems are purely mathematical, and have no connection
with the physical world.

\begin{definitionbox}
\textbf{Algebraic Vectors}

Let $n$ be any number in $\mathbb{N}$. An algebraic vector with $n$ components is an ordered set:
\[
\bar{x} = \langle x_1, x_2, ..., x_n \rangle, \quad \text{where } x_i \in \mathbb{R}
\]
These numbers are called the \textit{components} of $\bar{x}$.
\end{definitionbox}

\begin{definitionbox}
\textbf{Notation}

The set of all vectors with $n$ real-number components is denoted by $\mathbb{R}^n$.
\end{definitionbox}

\begin{examplebox}
\begin{itemize}
\item $\bar{x} = \langle 1, 2, -3 \rangle \in \mathbb{R}^3$
\item $\bar{y} = \langle -2, 0.7, 1.1, 0.2 \rangle \in \mathbb{R}^4$
\item $\bar{z} = \langle -1, 2, 3, -3 \rangle \in \mathbb{R}^4$
\end{itemize}
\end{examplebox}

\begin{definitionbox}
\textbf{Vector Equality}

Two vectors $\bar{x} = \langle x_1, x_2, ..., x_n \rangle$ and $\bar{y} = \langle y_1, y_2, ..., y_n \rangle$ are equal if and only if:
\[
x_1 = y_1, \quad x_2 = y_2, \quad ..., \quad x_n = y_n
\]
\end{definitionbox}

\begin{remarkbox}
Vectors are \textbf{ordered sets}. They are only equal if they have the same number of components and each corresponding component is equal.
\end{remarkbox}

\begin{examplebox}
The vectors $\bar{x} = \langle 1, 2, -3 \rangle$ and $\bar{y} = \langle 1, -3, 2 \rangle$ in $\mathbb{R}^3$ are not equal, since $x_3 = -3 \neq 2 = y_3$.

Also, let $\bar{w} \in \mathbb{R}^4$. Then the algebraic vectors $\bar{z} = \langle 1, 1, 1 \rangle$ and $\bar{w} = \langle 1, 1, 1, 1 \rangle$ are not equal, since $\bar{z} \in \mathbb{R}^3$ and $\bar{w} \in \mathbb{R}^4$.

\end{examplebox}

\newpage

% Section 2
\subsubsection{Algebraic Vector Operations}


The set $\mathbb{R}^n$ is equipped with algebraic operations in a natural way.
We can add vectors, multiply them by scalars, and subtract them.
These operations are defined as follows:
\begin{definitionbox}
\textbf{Vector Addition}

If $\bar{x}, \bar{y} \in \mathbb{R}^n$, then:
\[
\bar{x} + \bar{y} = \langle x_1 + y_1, x_2 + y_2, ..., x_n + y_n \rangle
\]
\end{definitionbox}

\begin{definitionbox}
\textbf{Scalar Vector Multiplication}

If $c \in \mathbb{R}$ and $\bar{x} \in \mathbb{R}^n$, then:
\[
c\bar{x} = \langle cx_1, cx_2, ..., cx_n \rangle
\]
\end{definitionbox}

\begin{notebox}
\textbf{Terminology...}

Some textbooks say that this is not the same as scalar \textit{product}, which refers 
to the dot product of two vectors. 
For our purposes we shall refer to the above definition as scalar vector \textit{multiplication}.
\end{notebox}

\begin{definitionbox}
\textbf{Zero Vector}

The \textit{zero vector} in $\mathbb{R}^n$ is:
\[
\bar{0} = \langle 0, 0, ..., 0 \rangle
\]
\end{definitionbox}

\begin{definitionbox}
\textbf{Additive Inverse of a Vector}

For $\bar{x} \in \mathbb{R}^n$, the additive inverse is:
\[
-\bar{x} = \langle -x_1, -x_2, ..., -x_n \rangle
\]
\end{definitionbox}

\begin{definitionbox}
\textbf{Vector Subtraction}

If $\bar{x}, \bar{y} \in \mathbb{R}^n$, then:
\[
\bar{x} - \bar{y} = \bar{x} + (-\bar{y})
\]
\end{definitionbox}

\begin{examplebox}
  Let $\bar{x} = \langle 2, 3, 9 \rangle$, $\bar{y} = \langle -2, 4, 1 \rangle$, $\bar{z} = \langle 2, 5, 1, 4 \rangle$ and $\bar{w} = \langle -4, 0, 2, -1 \rangle$.

\noindent Then:
\[
\bar{x} + \bar{y} = \langle 2, 3, 9 \rangle + \langle -2, 4, 1 \rangle 
= \langle 2 - 2, 3 + 4, 9 + 1 \rangle 
= \langle 0, 7, 10 \rangle,
\]
\[
\bar{z} + \bar{w} = \langle 2, 5, 1, 4 \rangle + \langle -4, 0, 2, -1 \rangle 
= \langle 2 - 4, 5 + 0, 1 + 2, 4 - 1 \rangle 
= \langle -2, 5, 3, 3 \rangle,
\]
\[
\text{and} \quad 7\bar{x} = 7 \langle 2, 3, 9 \rangle 
= \langle 7 \times 2, 7 \times 3, 7 \times 9 \rangle 
= \langle 14, 21, 63 \rangle.
\]

\noindent Note that $\bar{z} + \bar{w}$ is not defined, since $\bar{z} \in \mathbb{R}^4$ and $\bar{w} \in \mathbb{R}^3$.
\end{examplebox}

Vectors can be added, subtracted, and multiplied by scalars in a way that is 
consistent with the properties of real numbers.
We can summarise these properties in the following theorem:
\begin{theorembox}
\textbf{Properties of Vectors}

If $\bar{x}, \bar{y}, \bar{z} \in \mathbb{R}^n$ and $a, b \in \mathbb{R}$, then the following hold:
\begin{enumerate}
\item $\bar{x} + \bar{y} = \bar{y} + \bar{x}$ \hfill [Commutativity of Vector Addition]
\item $\bar{x} + (\bar{y} + \bar{z}) = (\bar{x} + \bar{y}) + \bar{z}$ \hfill [Associativity of Vector Addition]
\item $\bar{x} + \bar{0} = \bar{x}$ \hfill [Additive Identity of Vectors]
\item $\bar{x} + (-\bar{x}) = \bar{0}$ \hfill [Additive Inverse of a Vector]
\item $a(b\bar{x}) = (ab)\bar{x}$ \hfill [Associativity of Scalar Multiplication]
\item $1\bar{x} = \bar{x}$ \hfill [Multiplicative Identity]
\item $(-1)\bar{x} = -\bar{x}$ \hfill [Multiplication by $-1$]
\item $0\bar{x} = \bar{0}$ \hfill [Multiplication by a Zero Scalar]
\item $a\bar{0} = \bar{0}$ \hfill [Multiplying a Zero Vector]
\item $a(\bar{x} + \bar{y}) = a\bar{x} + a\bar{y}$ \hfill [Distributivity over Vector Addition]
\item $(a + b)\bar{x} = a\bar{x} + b\bar{x}$ \hfill [Distributivity over Scalar Addition]
\end{enumerate}
\end{theorembox}

%VectorPropertiesProofs.tex
% This file contains the proofs of the properties of vectors defined in WTW124.tex.

\begin{proofbox}
\begin{enumerate}[label=\arabic*., resume=vecprops]

\item \textbf{Commutativity of Vector Addition}

Assume $\bar{x}, \bar{y} \in \mathbb{R}^n$. 

We want to show that $\bar{x} + \bar{y} = \bar{y} + \bar{x}$.

\quad $\bar{x} + \bar{y} = \langle x_1 + y_1,\ x_2 + y_2,\ \ldots,\ x_n + y_n \rangle$ \hfill [Def. of Vector Addition]

\quad $= \langle y_1 + x_1,\ y_2 + x_2,\ \ldots,\ y_n + x_n \rangle$ \hfill [Commutativity of Real Numbers]

\quad $= \bar{y} + \bar{x}$ \hfill [Def. of Vector Addition]

\hfill $\qed$

\item \textbf{Associativity of Vector Addition}

Assume $\bar{x}, \bar{y}, \bar{z} \in \mathbb{R}^n$. 

We want to show that $\bar{x} + (\bar{y} + \bar{z}) = (\bar{x} + \bar{y}) + \bar{z}$.

\quad $\bar{x} + (\bar{y} + \bar{z}) = \bar{x} + \langle y_1 + z_1,\ y_2 + z_2,\ \ldots,\ y_n + z_n \rangle$ \hfill [Def. of Vector Addition]

\quad $= \langle x_1 + (y_1 + z_1),\ x_2 + (y_2 + z_2),\ \ldots,\ x_n + (y_n + z_n) \rangle$ \hfill [Def. of Vector Addition]

\quad $= \langle (x_1 + y_1) + z_1,\ (x_2 + y_2) + z_2,\ \ldots,\ (x_n + y_n) + z_n \rangle$ [Associativity of Real Numbers]

\quad $= \langle (x_1 + y_1),\ (x_2 + y_2),\ \ldots,\ (x_n + y_n) \rangle + \langle z_1,\ z_2,\ \ldots,\ z_n \rangle$ \hfill [Def. of Vector Addition]

\quad $= (\bar{x} + \bar{y}) + \bar{z}$ \hfill [Def. of Vector Addition]

\hfill $\qed$

\item \textbf{Additive Identity of Vectors}

Assume $\bar{x} \in \mathbb{R}^n$. 

We want to show that $\bar{x} + \bar{0} = \bar{x}$.

\quad $\bar{x} + \bar{0} = \langle x_1 + 0,\ x_2 + 0,\ \ldots,\ x_n + 0 \rangle$ \hfill [Def. of Vector Addition]

\quad $= \langle x_1,\ x_2,\ \ldots,\ x_n \rangle$ \hfill [Identity Property of Real Numbers]

\quad $= \bar{x}$ \hfill [Def. of Vector Equality]

\hfill $\qed$

\item \textbf{Additive Inverse of a Vector} 
 
Assume $\bar{x} \in \mathbb{R}^n$. 

We want to show that $\bar{x} + (-\bar{x}) = \bar{0}$.

\quad $\bar{x} + (-\bar{x}) = \langle x_1 + (-x_1),\ x_2 + (-x_2), \ \ldots, \ x_n + (-x_n) \rangle$ \hfill [Def. of Vector Addition]

\quad $= \langle 0, \ 0, \ \ldots, \ 0 \rangle$ \hfill [Additive Inverse of Real Numbers]

\quad $= \bar{0}$ \hfill [Def. of the Zero Vector]

\hfill $\qed$

\end{enumerate}
\end{proofbox}

\newpage

\begin{proofbox}
\begin{enumerate}[label=\arabic*., resume=vecprops]

\item \textbf{Associativity of Scalar Multiplication}

Assume $\bar{x} \in \mathbb{R}^n$.

We want to show that $a(b\bar{x}) = (ab)\bar{x}$.

\quad $a(b\bar{x}) = a\langle bx_1,\ bx_2,\ \ldots,\ bx_n \rangle$ \hfill [Def. of Scalar Vector Multiplication]

\quad $= \langle a(bx_1),\ a(bx_2),\ \ldots,\ a(bx_n) \rangle$ \hfill [Def. of Scalar Vector Multiplication]

\quad $= \langle (ab)x_1,\ (ab)x_2,\ \ldots,\ (ab)x_n \rangle$ \hfill [Associativity of Real Numbers]

\quad $= (ab)\langle x_1,\ x_2,\ \ldots,\ x_n \rangle$ \hfill [Def. of Scalar Vector Multiplication]

\quad $= (ab)\bar{x}$ \hfill [Def. of Vector Equality]

\hfill $\qed$

\item \textbf{Multiplicative Identity}

Assume $\bar{x} \in \mathbb{R}^n$.

We want to show that $1\bar{x} = \bar{x}$.

\quad $1\bar{x} = 1\langle x_1,\ x_2,\ \ldots,\ x_n \rangle$ \hfill [Def. of Scalar Vector Multiplication]

\quad $= \langle 1x_1,\ 1x_2,\ \ldots,\ 1x_n \rangle$ \hfill [Def. of Scalar Vector Multiplication]

\quad $= \langle x_1,\ x_2,\ \ldots,\ x_n \rangle$ \hfill [Identity Property of Real Numbers]

\quad $= \bar{x}$ \hfill [Def. of Vector Equality]

\hfill $\qed$

\item \textbf{Multiplication by $-1$}

Assume $\bar{x} \in \mathbb{R}^n$.

We want to show that $(-1)\bar{x} = -\bar{x}$.

\quad $(-1)\bar{x} = (-1)\langle x_1,\ x_2,\ \ldots,\ x_n \rangle$ \hfill [Def. of Scalar Vector Multiplication]

\quad $= \langle (-1)x_1,\ (-1)x_2,\ \ldots,\ (-1)x_n \rangle$ \hfill [Def. of Scalar Vector Multiplication]

\quad $= \langle -x_1,\ -x_2,\ \ldots,\ -x_n \rangle$ \hfill [Multiplication by $-1$ in Real Numbers]

\quad $= -\bar{x}$ \hfill [Def. of Additive Inverse of a Vector]

\hfill $\qed$

\item \textbf{Multiplication by a Zero Scalar}

Assume $\bar{x} \in \mathbb{R}^n$.

We want to show that $0\bar{x} = \bar{0}$.

\quad $0\bar{x} = 0\langle x_1,\ x_2,\ \ldots,\ x_n \rangle$ \hfill [Def. of Scalar Vector Multiplication]

\quad $= \langle 0x_1,\ 0x_2,\ \ldots,\ 0x_n \rangle$ \hfill [Def. of Scalar Vector Multiplication]

\quad $= \langle 0,\ 0,\ \ldots,\ 0 \rangle$ \hfill [Multiplication by Zero in Real Numbers]

\quad $= \bar{0}$ \hfill [Def. of the Zero Vector]

\hfill $\qed$

\end{enumerate}
\end{proofbox}

\newpage

\begin{proofbox}
\begin{enumerate}[label=\arabic*., resume=vecprops]

\item \textbf{Multiplying a Zero Vector}

Assume $\bar{x} \in \mathbb{R}^n$.

We want to show that $a\bar{0} = \bar{0}$.

\quad $a\bar{0} = a\langle 0,\ 0,\ \ldots,\ 0 \rangle$ \hfill [Def. of Scalar Vector Multiplication]

\quad $= \langle a \cdot 0,\ a \cdot 0,\ \ldots,\ a \cdot 0 \rangle$ \hfill [Def. of Scalar Vector Multiplication]

\quad $= \langle 0,\ 0,\ \ldots,\ 0 \rangle$ \hfill [Multiplication by Zero in Real Numbers]

\quad $= \bar{0}$ \hfill [Def. of the Zero Vector]

\hfill $\qed$

\item \textbf{Distributivity over Vector Addition}

Assume $\bar{x}, \bar{y} \in \mathbb{R}^n$ and $a \in \mathbb{R}$.

We want to show that $a(\bar{x} + \bar{y}) = a\bar{x} + a\bar{y}$.

\quad $a(\bar{x} + \bar{y}) = a\langle x_1 + y_1,\ x_2 + y_2,\ \ldots,\ x_n + y_n \rangle$ \hfill [Def. of Vector Addition]

\quad $= \langle a(x_1 + y_1),\ a(x_2 + y_2),\ \ldots,\ a(x_n + y_n) \rangle$ \hfill [Def. of Scalar Vector Multiplication]

\quad $= \langle ax_1 + ay_1,\ ax_2 + ay_2,\ \ldots,\ ax_n + ay_n \rangle$ \hfill [Distributivity of Real Numbers]

\quad $= \langle ax_1,\ ax_2,\ \ldots,\ ax_n \rangle + \langle ay_1,\ ay_2,\ \ldots,\ ay_n \rangle$ \hfill [Def. of Vector Addition]

\quad $= a\bar{x} + a\bar{y}$ \hfill [Def. of Scalar Vector Multiplication]

\hfill $\qed$

\item \textbf{Distributivity over Scalar Addition}

Assume $\bar{x} \in \mathbb{R}^n$ and $a, b \in \mathbb{R}$.

We want to show that $(a + b)\bar{x} = a\bar{x} + b\bar{x}$.

\quad $(a + b)\bar{x} = (a + b)\langle x_1,\ x_2,\ \ldots,\ x_n \rangle$ \hfill [Def. of Scalar Vector Multiplication]

\quad $= \langle (a + b)x_1,\ (a + b)x_2,\ \ldots,\ (a + b)x_n \rangle$ \hfill [Def. of Scalar Vector Multiplication]

\quad $= \langle ax_1 + bx_1,\ ax_2 + bx_2,\ \ldots,\ ax_n + bx_n \rangle$ \hfill [Distributivity of Real Numbers]

\quad $= \langle ax_1,\ ax_2,\ \ldots,\ ax_n \rangle + \langle bx_1,\ bx_2,\ \ldots,\ bx_n \rangle$ \hfill [Def. of Vector Addition]

\quad $= a\bar{x} + b\bar{x}$ \hfill [Def. of Scalar Vector Multiplication]

\hfill $\qed$

\end{enumerate}
\end{proofbox}


\begin{examplebox}
Let $\bar{x} = \langle -2, 3, 1 \rangle$, $\bar{y} = \langle 7, 0, 5 \rangle$, and $\bar{z} = \langle 4, 1, 8 \rangle$.

\vspace{1em}

Calculate $5(\bar{x} + 2\bar{y})$ and $\bar{x} - 3(\bar{x} + \bar{z})$

\vspace{1em}

\noindent\textbf{Solution.}
\begin{align*}
5(\bar{x} + 2\bar{y}) 
&= 5\bar{x} + 5(2\bar{y}) && \text{[Distributivity over vector addition]} \\
&= 5\bar{x} + 10\bar{y} \\
&= \langle -10, 15, 5 \rangle + \langle 70, 0, 50 \rangle && \text{[Definition of scalar multiplication]} \\
&= \langle 60, 15, 55 \rangle && \text{[Definition of vector addition]}
\end{align*}

\begin{align*}
\bar{x} - 3(\bar{x} + \bar{z}) 
&= \bar{x} - (3\bar{x} + 3\bar{z}) && \text{[Distributivity over vector addition]} \\
&= (\bar{x} - 3\bar{x}) - 3\bar{z} && \text{[Associativity of vector addition]} \\
&= -2\bar{x} - 3\bar{z} \\
&= \langle 4, -6, -2 \rangle + \langle -12, -3, -24 \rangle && \text{[Definition of scalar multiplication]} \\
&= \langle -8, -9, -26 \rangle && \text{[Definition of vector addition]}
\end{align*}
\end{examplebox}


In addition to the properties above, we can also define the \textbf{dot product} of two vectors, 
which is a way to multiply vectors that results in a scalar (real number).
\begin{definitionbox}
\textbf{Dot Product of Two Vectors}

Let $\bar{x} = \langle x_1, x_2, ..., x_n \rangle$ and $\bar{y} = \langle y_1, y_2, ..., y_n \rangle$ in $\mathbb{R}^n$. Then:
\[
\bar{x} \cdot \bar{y} = x_1 y_1 + x_2 y_2 + \cdots + x_n y_n
\]
\end{definitionbox}

\begin{remarkbox}
The dot product is only defined when $\bar{x}$ and $\bar{y}$ have the same number of components.
\end{remarkbox}

\begin{notebox}
Be careful: the scalar multiple of a vector is still a vector, but the dot product gives a real number.
\end{notebox}

\begin{examplebox}
Let $\bar{x} = \langle 2, 3, 9 \rangle$, $\bar{y} = \langle -2, 4, 1 \rangle$, $\bar{z} = \langle 2, 5, 1, 4 \rangle$, and $\bar{w} = \langle -4, 0, 2, -1 \rangle$.

\noindent Then
\[
\bar{x} \cdot \bar{y} 
= 2 \times (-2) + 3 \times 4 + 9 \times 1 
= -4 + 12 + 9 
= 17,
\]
and
\[
\bar{z} \cdot \bar{w} 
= 2 \times (-4) + 5 \times 0 + 1 \times 2 + 4 \times (-1) 
= -8 + 0 + 2 - 4 
= -10.
\]

\noindent Note that $\bar{x} \cdot \bar{z}$ is undefined, since $\bar{x} \in \mathbb{R}^3$ and $\bar{z} \in \mathbb{R}^4$.
\end{examplebox}

\begin{examplebox}
  \textbf{Scenario:}

Marti Stair is playing a game where he controls a drone moving in 3-dimensional space. At one moment, the drone's velocity vector is
\[
\bar{v} = \langle 3, -1, 4 \rangle,
\]
and the wind's velocity vector affecting the drone is
\[
\bar{w} = \langle 2, 5, -3 \rangle.
\]

Marti wants to find out how much of the drone's velocity is aligned with the wind's velocity by computing the dot product \(\bar{v} \cdot \bar{w}\).

\[
\bar{v} \cdot \bar{w} = (3)(2) + (-1)(5) + (4)(-3) = 6 - 5 - 12 = -11.
\]

Since the dot product \(\bar{v} \cdot \bar{w} = -11\) is negative, Marti concludes that the drone's velocity is generally moving against the direction of the wind's velocity vector; the wind is slowing the drone down in some directions.

\end{examplebox}
The dot product has its own properties that make it useful in various applications, such as physics and computer graphics.
We can use it to find angles between vectors, project one vector onto another, and more.
\begin{theorembox}
\textbf{Properties of the Dot Product}

If $\bar{x}, \bar{y}, \bar{z} \in \mathbb{R}^n$, and $a, b \in \mathbb{R}$ then the following hold:
\begin{enumerate}
\item \textbf{Positive Definiteness:}
\begin{itemize}
\item $\bar{x} \cdot \bar{x} \geq 0$
\item $\bar{x} \cdot \bar{x} = 0 \Leftrightarrow \bar{x} = \bar{0}$
\end{itemize}
\item \textbf{Commutativity of the Dot Product:} $\bar{x} \cdot \bar{y} = \bar{y} \cdot \bar{x}$
\item \textbf{Distributivity of the Dot Product:} $\bar{x} \cdot (a\bar{y} + b\bar{z}) = a(\bar{x} \cdot \bar{y}) + b(\bar{x} \cdot \bar{z})$
\end{enumerate}
\end{theorembox}

%DotProductProofs.tex
% This file contains the proofs for the properties of the dot product.

\begin{proofbox}
\begin{enumerate}[label=\arabic*., series=vecprops]

\item \textbf{Positive Definiteness}

Assume $\bar{x} \in \mathbb{R}^n$.

We shall first show that $\bar{x} \cdot \bar{x} \geq 0$.

\quad $\bar{x} \cdot \bar{x} = (x_1)^2 + (x_2)^2 + \cdots + (x_n)^2$ \hfill [Def. of Dot Product]

\quad $= \sum_{i=1}^n (x_i)^2$ \hfill [Summation Notation]

\quad $\geq 0$ \hfill $[ \because(x_i)^2 \geq 0 \text{ for all } i \in \{1, 2, \ldots, n\}]$


Now, we shall show that $\bar{x} \cdot \bar{x} = 0 \Leftrightarrow \bar{x} = \bar{0}$.

Suppose $\bar{x} \cdot \bar{x} = 0$.

\quad $\bar{x} \cdot \bar{x} = (x_1)^2 + (x_2)^2 + \cdots + (x_n)^2 = 0$ \hfill [Def. of Dot Product]

\quad $\Rightarrow for\ all\ i \in \{1, 2, \ldots, n\},\ (x_i)^2 = 0$ \hfill [Each term must be zero]

Now for contradiction, assume $\bar{x} \neq \bar{0}$.

\quad $\Rightarrow \exists i \in \{1, 2, \ldots, n\} \text{ such that } x_i \neq 0$.

\quad $\Rightarrow (x_i)^2 > 0$ \hfill [Square of a non-zero number is positive]

\quad This contradicts $(x_i)^2 = 0$.

\quad $\therefore \bar{x} = \bar{0}$.

Conversely, if $\bar{x} = \bar{0}$, then:

\quad $\bar{x} \cdot \bar{x} = 0^2 + 0^2 + \cdots + 0^2 = 0$ \hfill [Def. of Dot Product]

\quad $\therefore \bar{x} \cdot \bar{x} = 0$.

\hfill $\qed$

\item \textbf{Commutativity of the Dot Product}

Assume $\bar{x}, \bar{y} \in \mathbb{R}^n$.

We want to show that $\bar{x} \cdot \bar{y} = \bar{y} \cdot \bar{x}$.

\quad $\bar{x} \cdot \bar{y} = x_1 y_1 + x_2 y_2 + \cdots + x_n y_n$ \hfill [Def. of Dot Product]

\quad $= y_1 x_1 + y_2 x_2 + \cdots + y_n x_n$ \hfill [Commutativity of Real Numbers]

\quad $= \bar{y} \cdot \bar{x}$ \hfill [Def. of Dot Product]

\hfill $\qed$

\item \textbf{Distributive Law of the Dot Product}

Assume $\bar{x}, \bar{y}, \bar{z} \in \mathbb{R}^n$ and $a, b \in \mathbb{R}$.

We want to show that $\bar{x} \cdot (a\bar{y} + b\bar{z}) = a(\bar{x} \cdot \bar{y}) + b(\bar{x} \cdot \bar{z})$.

\quad $\bar{x} \cdot (a\bar{y} + b\bar{z}) = \bar{x} \cdot \langle ay_1 + bz_1, ay_2 + bz_2, \ldots, ay_n + bz_n \rangle$ \hfill [Def. of Vector Addition]

\quad $= x_1(ay_1 + bz_1) + x_2(ay_2 + bz_2) + \cdots + x_n(ay_n + bz_n)$ \hfill [Def. of Dot Product]

\quad $= a(x_1 y_1 + x_2 y_2 + \cdots + x_n y_n) + b(x_1 z_1 + x_2 z_2 + \cdots + x_n z_n)$ [Distributivity over $\mathbb{R}$]

\quad $= a(\bar{x} \cdot \bar{y}) + b(\bar{x} \cdot \bar{z})$ \hfill [Def. of Dot Product]

\hfill $\qed$

\end{enumerate}
\end{proofbox}


\begin{examplebox}
Let $\bar{x} = \langle -2,3,1 \rangle$, $\bar{y} = \langle 7,0,5 \rangle$, and $\bar{z} = \langle 4,1,8 \rangle$. We calculate
$\bar{x} \cdot (2\bar{y} - \bar{z})$ and $(\bar{x} + \bar{y}) \cdot (\bar{x} + 2\bar{z})$.

\smallskip

\quad $\bar{x} \cdot (2\bar{y} - \bar{z})$ 

\quad $= 2(\bar{x} \cdot \bar{y}) - (\bar{x} \cdot \bar{z})$ \hfill [Distributivity over Vector Addition]

\quad $= 2((-2)(7) + 3(0) + 1(5)) - ((-2)(4) + 3(1) + 1(8))$ \hfill [Definition of Dot Product]

\quad $= 2(-14 + 0 + 5) - (-8 + 3 + 8)$ \hfill [Simplify]

\quad $= 2(-9) - 3 = -18 - 3 = -21$.

\vspace{1em}

\quad $(\bar{x} + \bar{y}) \cdot (\bar{x} + 2\bar{z})$

\quad $= (\bar{x} + \bar{y}) \cdot \bar{x} + 2((\bar{x} + \bar{y}) \cdot \bar{z})$ \hfill [Distributivity over Vector Addition] 

\quad $= \bar{x} \cdot (\bar{x} + \bar{y}) + 2(\bar{z} \cdot (\bar{x} + \bar{y}))$ \hfill [Commutativity of Vector Addition]

\quad $= \bar{x} \cdot \bar{x} + \bar{x} \cdot \bar{y} + 2(\bar{z} \cdot \bar{x}) + 2(\bar{z} \cdot \bar{y})$ \hfill [Distributivity of Dot Product]

\quad $=14 - 9 + 6 + 136 = 147$.

\end{examplebox}

With these properties, we can also define the \textbf{norm} of a vector, 
which is a measure of its length or magnitude.
\begin{definitionbox}
\textbf{Norm of a Vector}

The norm of a vector $\bar{x} = \langle x_1, x_2, ..., x_n \rangle$ is:
\[
\|\bar{x}\| = \sqrt{\bar{x} \cdot \bar{x}} = \sqrt{x_1^2 + x_2^2 + \cdots + x_n^2}
\]
\end{definitionbox}

\begin{notebox}
\begin{enumerate}
\item By the dot product properties, the norm $\|\bar{x}\|$ is well-defined for every $\bar{x} \in \mathbb{R}^n$.
\item If a vector $\bar{u}$ in $\mathbb{R}^n$ has $\|\bar{u}\| = 1$, then it's a \textbf{unit vector}.
\end{enumerate}

In $\mathbb{R}^3$, the standard unit vectors are:
\[
\bar{i} = \langle 1, 0, 0 \rangle, \quad \bar{j} = \langle 0, 1, 0 \rangle, \quad \bar{k} = \langle 0, 0, 1 \rangle
\]
The set $\{\bar{i}, \bar{j}, \bar{k}\}$ is the \textbf{standard basis} for $\mathbb{R}^3$. 
Any vector $\bar{x} = \langle x_1, x_2, x_3 \rangle$ can be written as:
\[
\bar{x} = x_1 \bar{i} + x_2 \bar{j} + x_3 \bar{k}
\]

Think of it like a set of coordinate translation factors!
\end{notebox}

\begin{examplebox}
  Let $\bar{x} = \langle 1,2,-3 \rangle$, $\bar{y} = \langle -2,0,7 \rangle$, and $\bar{z} = \langle -1,2,3,-3 \rangle$. Then

\quad $\|\bar{x}\| = \sqrt{\bar{x} \cdot \bar{x}} = \sqrt{1^2 + 2^2 + (-3)^2} = \sqrt{1 + 4 + 9} = \sqrt{14}$

\quad $\|\bar{y}\| = \sqrt{\bar{y} \cdot \bar{y}} = \sqrt{(-2)^2 + 0^2 + 7^2} = \sqrt{4 + 0 + 49} = \sqrt{53}$

\quad $\|\bar{z}\| = \sqrt{\bar{z} \cdot \bar{z}} = \sqrt{(-1)^2 + 2^2 + 3^2 + (-3)^2} = \sqrt{1 + 4 + 9 + 9} = \sqrt{23}$

\end{examplebox}

One of the most important properties of the norm of an algebraic vector, in relation to the
dot product, is the following result, known as the Cauchy-Schwarz Inequality.

\begin{theorembox}
\textbf{Cauchy-Schwarz Inequality}

If $\bar{x}, \bar{y} \in \mathbb{R}^n$, then:

\[
|\bar{x} \cdot \bar{y}| \leq \|\bar{x}\| \|\bar{y}\|
\]
\end{theorembox}

%CauchySchwarzIneq.tex
% This file contains the proof of the Cauchy-Schwarz Inequality.

\begin{proofbox}
\begingroup
\setlength{\baselineskip}{1.5\baselineskip} 

Assume $\bar{x}, \bar{y} \in \mathbb{R}^n$.

We want to show that $|\bar{x} \cdot \bar{y}| \leq \|\bar{x}\| \cdot \|\bar{y}\|$.

\textbf{Case 1: Either $\bar{x} = \bar{0}$ or $\bar{y} = \bar{0}$.}

\quad Without loss of generality, assume $\bar{y} = \bar{0}$.

\quad Then $\|\bar{y}\| = \sqrt{\bar{y} \cdot \bar{y}} = \sqrt{0} = 0$ \hfill [Def. of Norm of a Vector].

\quad And $|\bar{x} \cdot \bar{y}| = |\sum_{i=1}^n (x_i \cdot 0)| = 0$ \hfill [Def. of Dot Product].

\quad $\therefore |\bar{x} \cdot \bar{y}| = 0 = \|\bar{y}\|$.

\quad This implies $|\bar{x} \cdot \bar{y}| = \|\bar{x}\| \|\bar{y}\|$.

\quad Furthermore, since $\|\bar{y}\| = 0$, we have $|\bar{x} \cdot \bar{y}| \leq \|\bar{x}\| \|\bar{y}\|$.

\textbf{Case 2: $\bar{y} \neq \bar{0}$.}

\quad $||\bar{y}||^2 = \bar{y} \cdot \bar{y} > 0$. \hfill [Def. of Norm of a Vector]

\quad $\Rightarrow \exists a \in \mathbb{R}$ such that $a = \frac{\bar{x} \cdot \bar{y}}{\|\bar{y}\|^2}$.

\quad Now consider $(\bar{x} - a \bar{y}) \cdot (\bar{x} - a \bar{y})$.

\quad By the properties of the dot product, we have:

\quad $(\bar{x} - a \bar{y}) \cdot (\bar{x} - a \bar{y}) = \bar{x} \cdot \bar{x} - 2a(\bar{x} \cdot \bar{y}) + a^2(\bar{y} \cdot \bar{y})$.

\quad $= ||\bar{x}||^2 - 2a(\bar{x} \cdot \bar{y}) + a^2 ||\bar{y}||^2$.

\quad $= ||\bar{x}||^2 - 2 \frac{(\bar{x} \cdot \bar{y})^2}{||\bar{y}||^2} + \frac{(\bar{x} \cdot \bar{y})^2}{||\bar{y}||^2}$ \hfill [Substituting $a$].

\quad $= ||\bar{x}||^2 - \frac{(\bar{x} \cdot \bar{y})^2}{||\bar{y}||^2}$. \hfill [Combining terms].

Since the dot product is positive definite, we know that 

\quad $(\bar{x} - a\bar{y}) \cdot (\bar{x} - a\bar{y}) \geq 0$.

\quad Therefore:

\quad $||\bar{x}||^2 - \frac{(\bar{x} \cdot \bar{y})^2}{||\bar{y}||^2} \geq 0$

\quad $\Rightarrow ||\bar{x}||^2 \cdot ||\bar{y}||^2 \geq (\bar{x} \cdot \bar{y})^2$

\quad $\Rightarrow ||\bar{x}|| \cdot ||\bar{y}|| \geq |\bar{x} \cdot \bar{y}|$

\quad Thus, the Cauchy-Schwarz inequality holds in this case as well.

\hfill \qed

\endgroup
\end{proofbox}



Let $\bar{x}$ and $\bar{y}$ be algebraic vectors in $\mathbb{R}^n$. 
According to the Cauchy-Schwarz Inequality,

$|\bar{x} \cdot \bar{y}| \leq \|\bar{x}\|\|\bar{y}\|$.

This inequality includes two possibilities; namely,
$|\bar{x} \cdot \bar{y}| < \|\bar{x}\|\|\bar{y}\|$ or $|\bar{x} \cdot \bar{y}| = \|\bar{x}\|\|\bar{y}\|$.

In applications, it is important to know when equality holds in the Cauchy-Schwarz 

Inequality; that is, when $|\bar{x} \cdot \bar{y}| = \|\bar{x}\|\|\bar{y}\|$. 
We can thus make the following claim:
\begin{propositionbox}
 $|\bar{x} \cdot \bar{y}| = \|\bar{x}\|\|\bar{y}\|$ if and only if $\bar{y} = a\bar{x}$ or $\bar{x} = a\bar{y}$ for some $a \in \mathbb{R}$.
 
\end{propositionbox}

\begin{proofbox}
First, assume that \( |\bar{x} \cdot \bar{y}| = \|\bar{x}\| \|\bar{y}\| \).  
We want to show that \( \bar{x} = \alpha \bar{y} \) for some \( \alpha \in \mathbb{R} \), or \( \bar{x} = \bar{0} \).

If \( \bar{y} = \bar{0} \), then both sides of the equation equal zero, and the equality holds trivially.

Suppose \( \bar{y} \neq \bar{0} \). Then \( \|\bar{y}\| > 0 \), and we define the real number
\[
\alpha = \frac{\bar{x} \cdot \bar{y}}{\|\bar{y}\|^2}.
\]
Then,
\[
\bar{x} \cdot \bar{y} = \alpha \|\bar{y}\|^2.
\]
Taking absolute values:
\[
|\bar{x} \cdot \bar{y}| = |\alpha| \|\bar{y}\|^2.
\]
But by assumption,
\[
|\bar{x} \cdot \bar{y}| = \|\bar{x}\| \|\bar{y}\| \Rightarrow |\alpha| \|\bar{y}\|^2 = \|\bar{x}\| \|\bar{y}\|.
\]
Divide both sides by \( \|\bar{y}\| \) (since it's nonzero):
\[
|\alpha| \|\bar{y}\| = \|\bar{x}\|.
\]
Now square both sides:
\[
\alpha^2 \|\bar{y}\|^2 = \|\bar{x}\|^2.
\]
But also,
\[
\|\bar{x}\|^2 = \bar{x} \cdot \bar{x}, \quad \text{and} \quad \alpha^2 \|\bar{y}\|^2 = (\alpha \bar{y}) \cdot (\alpha \bar{y}).
\]

Hence,
\(
\bar{x} \cdot \bar{x} = (\alpha \bar{y}) \cdot (\alpha \bar{y}) \Rightarrow \|\bar{x} - \alpha \bar{y}\|^2 = 0,
\)
which implies that \(
\bar{x} = \alpha \bar{y}.
\)
\vspace{1em}

Now assume that \( \bar{x} = \alpha \bar{y} \) for some \( \alpha \in \mathbb{R} \), and \( \bar{y} \neq \bar{0} \). We want to show that
\(
|\bar{x} \cdot \bar{y}| = \|\bar{x}\| \|\bar{y}\|.
\)

We compute:
\(
|\bar{x} \cdot \bar{y}| = |\alpha \bar{y} \cdot \bar{y}| = |\alpha| |\bar{y} \cdot \bar{y}| = |\alpha| \|\bar{y}\|^2.
\)

Also,
\(
\|\bar{x}\| = \|\alpha \bar{y}\| = |\alpha| \|\bar{y}\| \Rightarrow \|\bar{x}\| \|\bar{y}\| = |\alpha| \|\bar{y}\|^2.
\)

Therefore,
\(
|\bar{x} \cdot \bar{y}| = \|\bar{x}\| \|\bar{y}\|.
\)

\hfill $\qed$

\end{proofbox}

Using the properties of the dot product and the Cauchy-Schwarz Inequality, we can obtain properties of the norm of a vector.

\begin{theorembox}

\textbf{Properties of the Norm of a Vector}

If $\bar{x}, \bar{y} \in \mathbb{R}^n$, and $a \in \mathbb{R}$ then the following hold:

\begin{enumerate}

\item $\|\bar{x}\| \geq 0$, and $\|\bar{x}\| = 0 \Leftrightarrow \bar{x} = \bar{0}$ \hfill [Positive Definiteness]
\item $\|a\bar{x}\| = |a|\|\bar{x}\|$ \hfill [Multiplicative Property]
\item $\|\bar{x} + \bar{y}\| \leq \|\bar{x}\| + \|\bar{y}\|$ \hfill [Triangle Inequality]

\end{enumerate}
\end{theorembox}


\begin{proofbox}
\begin{enumerate}[label=\arabic*., series=normprops]
    \item \textbf{Positive Definiteness}
    
    Assume $\bar{x} \in \mathbb{R}^n$. We shall first show that $\|\bar{x}\| \geq 0$.

    \quad $\|\bar{x}\| = \sqrt{x_1^2 + x_2^2 + \cdots + x_n^2}$ \hfill [Def. of Norm]

    \quad $= \sqrt{\sum_{i=1}^n (x_i)^2}$ \hfill [Summation Notation]

    \quad $\geq 0$ \hfill $[ \because (x_i)^2 \geq 0 \text{ for all } i \in \{1, 2, \ldots, n\}]$

    Now, we shall show that $\|\bar{x}\| = 0 \Leftrightarrow \bar{x} = \bar{0}$.
    Suppose $\|\bar{x}\| = 0$.

    \quad $\|\bar{x}\| = \sqrt{x_1^2 + x_2^2 + \cdots + x_n^2} = 0$ \hfill [Def. of Norm]

    \quad $\Rightarrow x_1^2 + x_2^2 + \cdots + x_n^2 = 0$ \hfill [Square root is zero]

    \quad $\Rightarrow (x_i)^2 = 0$ for all $i \in \{1, 2, \ldots, n\}$ \hfill [Each term must be zero]

    Now for contradiction, assume $\bar{x} \neq \bar{0}$.

    \quad $\Rightarrow \exists i \in \{1, 2, \ldots, n\} \text{ such that } x_i \neq 0$.

    \quad $\Rightarrow (x_i)^2 > 0$ \hfill [Square of a non-zero number is positive]

    \quad This contradicts $(x_i)^2 = 0$.

    \quad $\therefore \bar{x} = \bar{0}$.

    Conversely, if $\bar{x} = \bar{0}$, then:

    \quad $\|\bar{x}\| = \sqrt{0^2 + 0^2 + \cdots + 0^2} = 0$ \hfill [Def. of Norm]

    \quad $\therefore \|\bar{x}\| = 0$.

    \hfill $\qed$
\end{enumerate}

\end{proofbox}

\newpage

\begin{proofbox}
\begin{enumerate}[label=\arabic*., resume=normprops]


\item \textbf{Multiplicative Property}
    
    Assume $\bar{x} \in \mathbb{R}^n$ and $c \in \mathbb{R}$.

    We want to show that $\|c \bar{x}\| = |c| \|\bar{x}\|$.

    \quad $\|c \bar{x}\| = \sqrt{(c x_1)^2 + (c x_2)^2 + \cdots + (c x_n)^2}$ \hfill [Def. of Norm]

    \quad $= \sqrt{c^2 (x_1^2 + x_2^2 + \cdots + x_n^2)}$ \hfill [Factoring out $c^2$]

    \quad $= |c| \sqrt{x_1^2 + x_2^2 + \cdots + x_n^2}$ \hfill [Square root of $c^2$ is $|c|$]

    \quad $= |c| \|\bar{x}\|$ \hfill [Def. of Norm]

    \hfill $\qed$

    \item \textbf{Triangle Inequality}
    
    Assume $\bar{x}, \bar{y} \in \mathbb{R}^n$.

    We want to show that $\|\bar{x} + \bar{y}\| \leq \|\bar{x}\| + \|\bar{y}\|$.

    By the Cauchy-Schwarz inequality, we have:

    \quad $\|\bar{x} + \bar{y}\|^2 = \sum_{i=1}^n (x_i + y_i)^2$ \hfill [Def. of Norm]

    \quad $= \sum_{i=1}^n (x_i^2 + 2 x_i y_i + y_i^2)$ \hfill [Expanding the square]

    \quad $= \sum_{i=1}^n x_i^2 + 2 \sum_{i=1}^n x_i y_i + \sum_{i=1}^n y_i^2$ \hfill [Distributing the summation]

    \quad $= \|\bar{x}\|^2 + 2 \sum_{i=1}^n x_i y_i + \|\bar{y}\|^2$ \hfill [Def. of Norm]

    By the Cauchy-Schwarz inequality, we know:

    \quad $\left( \sum_{i=1}^n x_i y_i \right)^2 \leq \left( \sum_{i=1}^n x_i^2 \right) \left( \sum_{i=1}^n y_i^2 \right)$

    \quad $\Rightarrow 2 \sum_{i=1}^n x_i y_i \leq 2 \sqrt{\left( \sum_{i=1}^n x_i^2 \right) \left( \sum_{i=1}^n y_i^2 \right)}$

    \quad $\Rightarrow 2 \sum_{i=1}^n x_i y_i \leq 2 \|\bar{x}\| \|\bar{y}\|$ \hfill [Def. of Norm]

    Thus, we have:

    \quad $\|\bar{x} + \bar{y}\|^2 \leq \|\bar{x}\|^2 + 2 \|\bar{x}\| \|\bar{y}\| + \|\bar{y}\|^2$

    \quad $= (\|\bar{x}\| + \|\bar{y}\|)^2$ \hfill [Factoring the right-hand side]

    Taking the square root of both sides gives:

    \quad $\|\bar{x} + \bar{y}\| \leq \|\bar{x}\| + \|\bar{y}\|$

    \hfill $\qed$
    
\end{enumerate}
\end{proofbox}

Now that we have established the basic properties of algebraic vectors in $\mathbb{R}^n$, we can apply these concepts to model
three-dimensional space using the algebraic structure of $\mathbb{R}^3$.

% ExercisesUnit1-1.tex
% This file contains exercises for Unit 1-1 of WTW124.

\begin{exercisebox}
\begin{enumerate}[label=\arabic*., series=exercises]

\item Let $\bar{v} = \langle 3, -6, 7 \rangle$, $\bar{x} = \langle 2, 1, 2 \rangle$, $\bar{y} = \langle -1, 8, 1 \rangle$,\\ 
$\bar{z} = \langle -2, 3, 0, 2 \rangle$, $\bar{w} = \langle 9, -2, 1, 1 \rangle$.

Calculate each of the following algebraic vectors, if it is defined. If it is not defined, explain why.

\begin{multicols}{3}
\begin{enumerate}[label=(\alph*)]
\item $5\bar{x} - 2\bar{y}$
\item $\bar{v} + 6(\bar{y} - \bar{x})$
\item $\bar{z} - 2(\bar{x} + \bar{y})$
\item $2\bar{x} - 7(\bar{v} + 3\bar{y})$
\item $2 + \bar{x}$
\item $3\bar{z} - 2(\bar{w} + \bar{z})$
\item $6(3\bar{x} + \bar{y} - 2\bar{v})$
\item $7\bar{y} - 2\bar{x} + 3\bar{v}$
\item $\bar{x} + (\bar{v} - \bar{w})$
\item $\bar{x} + 0\bar{y} - 2\bar{v}$
\end{enumerate}
\end{multicols}


\item Let $\bar{x} = \langle 1, \alpha, -2 \rangle$, $\bar{y} = \langle \beta, 1 - \beta, \alpha \rangle$, and $\bar{z} = \langle 1, 8, -1 \rangle$,\\
where $\alpha$ and $\beta$ are real numbers. Find all values of $\alpha$ and $\beta$, if any, for which the following equations are true:
\begin{multicols}{3}
\begin{enumerate}[label=(\alph*)]
\item $2\bar{x} + 3\bar{y} = \bar{z}$
\item $\bar{x} - \bar{y} = \bar{0}$
\item $\alpha \bar{x} + 2\bar{y} = \langle 7, -3, 0 \rangle$
\end{enumerate}
\end{multicols}

\item If $\bar{a}$, $\bar{b}$ and $\bar{c}$ are algebraic vectors in $\mathbb{R}^n$, then $\bar{a}$ is a \textit{linear combination} of $\bar{b}$ and $\bar{c}$ if there exist real numbers $\alpha$ and $\beta$ such that:
\[
\bar{a} = \alpha \bar{b} + \beta \bar{c}.
\]
Let $\bar{b} = \langle -1, 2, 1 \rangle$ and $\bar{c} = \langle 1, 1, 1 \rangle$.

Determine whether the following vectors are linear combinations of $\bar{b}$ and $\bar{c}$:
\begin{multicols}{3}
\begin{itemize}
\item $\bar{p} = \langle 2, 5, 4 \rangle$
\item $\bar{q} = \langle -4, 2, 0 \rangle$
\item $\bar{r} = \langle 2, -4, -1 \rangle$
\end{itemize}
\end{multicols}

\item Use properties of vectors to prove the following:

If $\bar{x}, \bar{y}, \bar{z} \in \mathbb{R}^n$ such that $\bar{x} + \bar{z} = \bar{y} + \bar{z}$, then $\bar{x} = \bar{y}$.

\item Prove that if $\bar{x}, \bar{y} \in \mathbb{R}^3$ and $\alpha \in \mathbb{R}$ with $\alpha \neq 0$ such that $\alpha \bar{x} = \alpha \bar{y}$, then $\bar{x} = \bar{y}$.

\item Let $\bar{v} = \langle 3,-6,7 \rangle$, $\bar{x} = \langle 2,1,2 \rangle$, $\bar{y} = \langle -1,8,1 \rangle$, $\bar{z} = \langle -2,3,0,2 \rangle$, $\bar{w} = \langle 9,-2,1,1 \rangle$. \\
Calculate the following, if possible. Otherwise, explain why it is not possible to evaluate the given expression.

\begin{multicols}{2}
\begin{enumerate}[label=(\alph*)]
\item $\|\ 5\bar{x} - 2\bar{y} \|$
\item $\bar{v} \cdot (\bar{y} - 2\bar{x})$
\item $\|\ \bar{z} - \bar{x} + \bar{y} \|$
\item $(\bar{w} - 2\bar{z}) \cdot (\bar{w} + 2\bar{z})$
\item $\| \bar{x} \cdot \bar{y} \|$
\item $(2\bar{v} - \bar{x} + 3\bar{y}) \cdot (\bar{x} - \bar{v})$
\item $(\|\bar{x}\|\bar{y} - \|\bar{y}\|\bar{x}) \cdot (\|\bar{x}\|\bar{y} - \|\bar{y}\|\bar{x})$
\item $\bar{x} \cdot (\bar{v} - \bar{y})$
\item $(2\bar{x}) \cdot \bar{y} + \bar{v}$
\item $\| 7\bar{y} - 2\bar{x} + 3\bar{v} \|$
\end{enumerate}
\end{multicols}

\end{enumerate}
\end{exercisebox}

\newpage

\begin{exercisebox}
\begin{enumerate}[label=\arabic*., resume=exercises]
    
\item Let $\bar{x}$ and $\bar{y}$ be algebraic vectors in $\mathbb{R}^3$.
\begin{enumerate}[label=(\alph*)]
\item If $\bar{x} = \alpha \bar{y}$ for some $\alpha \in \mathbb{R}$, show that $|\bar{x} \cdot \bar{y}| = \|\bar{x}\|\|\bar{y}\|$.
\item Now suppose $|\bar{x} \cdot \bar{y}| = \|\bar{x}\|\|\bar{y}\|$, and $\bar{y} \ne \bar{0}$. In the proof of the Cauchy–Schwarz inequality it is shown that:
\[
0 \le (\bar{x} - \alpha \bar{y}) \cdot (\bar{x} - \alpha \bar{y}) = \|\bar{x}\|^2 - \frac{(\bar{x} \cdot \bar{y})^2}{\|\bar{y}\|^2}
\]
for some real number $\alpha$. Use this fact to prove that $\bar{x} = \alpha \bar{y}$.
\end{enumerate}

\item Let $\bar{u}$ and $\bar{v}$ be algebraic vectors in $\mathbb{R}^3$. Prove the following:
\begin{enumerate}[label=(\alph*)]
\item $\|\bar{u} - \bar{v}\|^2 + \|\bar{u} + \bar{v}\|^2 = 2\|\bar{u}\|^2 + 2\|\bar{v}\|^2$.
\item $\bar{u} \cdot \bar{v} = \frac{1}{4} \left( \|\bar{u} + \bar{v}\|^2 - \|\bar{u} - \bar{v}\|^2 \right)$.
[Hint: Use the definition of the norm, and the properties of the dot product.]
\end{enumerate}
\end{enumerate}
\end{exercisebox}



\newpage
  
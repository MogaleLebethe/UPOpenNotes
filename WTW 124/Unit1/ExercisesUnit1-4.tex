\begin{exercisebox}
\begin{enumerate}[series=AngleEX]
  \item Find the magnitude of the given angle, if it is defined. Otherwise, explain why it is not defined.
  \begin{enumerate}
    \item The angle determined by the rays with equations $\bar{x} = \bar{0} + t\langle 0, 2, 1 \rangle$ and $\bar{x} = \bar{0} + t\langle 3, -1, 2 \rangle$, $t \geq 0$
    \item The angle determined by the rays with equations $\bar{x} = \langle 1, 0, 1 \rangle + t\langle 0, 2, 0 \rangle$ and $\bar{x} = \langle 1, 0, 1 \rangle + t\langle -1, 3, \sqrt{2} \rangle$, $t \geq 0$
    \item The angle determined by the rays with equations $\bar{x} = \langle 1, 2, 3 \rangle + t\langle \sqrt{2}, -3\sqrt{2}, -2 \rangle$ and $\bar{x} = \langle 1, 2, 3 \rangle + t\langle 0, 2, 0 \rangle$, $t \geq 0$
    \item The angle determined by the rays with equations $\bar{x} = \langle -2, 2, 1 \rangle + t\langle 4, 8, 4 \rangle$ and $\bar{x} = \langle -2, 2, 1 \rangle + t\langle 1, 2, 1 \rangle$, $t \geq 0$
    \item The angle determined by the rays with equations $\bar{x} = \langle 3, 2, 1 \rangle + t\langle 1, 0, \frac{1}{\sqrt{3}} \rangle$ and $\bar{x} = \langle 2, 2, 5 \rangle + t\langle 0, 0, -1 \rangle$, $t \geq 0$
    \item The angle determined by the rays with equations $\bar{x} = \langle 2, 2, 5 \rangle + t\langle 3, 0, \sqrt{3} \rangle$ and $\bar{x} = \langle 2, 2, 5 \rangle + t\langle 0, 0, 2 \rangle$, $t \geq 0$
    \item The angle determined by the rays with equations $\bar{x} = \langle 0, -1, -3 \rangle + t\langle 1+\sqrt{3}, 2, 1 - \sqrt{\sqrt{3}} \rangle$ and $\bar{x} = \langle 0, -1, -3 \rangle + t\langle 1, 2, 1 \rangle$, $t \geq 0$
    \item The angle determined by the rays with equations $\bar{x} = \langle 0, -1, -3 \rangle + t\langle -1, 1 + 3, 2 + \sqrt{3} \rangle$ and $\bar{x} = \langle 0, -1, -3 \rangle + t\langle -2, -4, -2 \rangle$, $t \geq 0$
    \item The angle determined by the rays with equations $\bar{x} = \langle 1, 2, 1 \rangle + t\langle 1, 1, 2 \rangle$ and $\bar{x} = \langle 1, 2, 1 \rangle + t\langle -1, -1, 2 \rangle$, $t \geq 0$
  \end{enumerate}

  \item Consider the triangle with vertices $\bar{a}, \bar{b}, \bar{c}$. In each case, find the magnitude of the specified angle.
  \begin{enumerate}
    \item $\bar{a} = \langle 1,1,1 \rangle$, $\bar{b} = \langle 1,3,2 \rangle$, $\bar{c}= \langle 7,-1,5 \rangle$; the angle at $\bar{a}$
    \item $\bar{a} = \langle 1,-1,1 \rangle$, $\bar{b} = \langle 1,3,1 \rangle$, $\bar{c}= \langle 1 - \sqrt{2}, 3\sqrt{2} - 1, 3 \rangle$; the angle at $\bar{a}$
    \item $\bar{a} = \langle 1,1,1 \rangle$, $\bar{b} = \langle 0,1,0 \rangle$, $\bar{c}= \langle 0,0,1 \rangle$; the angle at $\bar{c}$
    \item $\bar{a} = \bar{0}$, $\bar{b} = \langle 2, -6, -2\sqrt{2} \rangle$, $\bar{c}= \langle 0, 1, 0 \rangle$; the angle at $\bar{a}$
    \item $\bar{a} = \langle \sqrt{3}, 3, -\sqrt{3} \rangle$, $\bar{b} = \langle 1, 5, 1 \rangle$, $\bar{c} = \langle -1, 1, -1 \rangle$; the angle at $\bar{c}$
    \item $\bar{a} = \langle 0, 2, -3 \rangle$, $\bar{b} = \langle \sqrt{3}, 2, -2 \rangle$, $\bar{c}= \langle 0, 2, -5 \rangle$; the angle at $\bar{a}$
    \item $\bar{a} = \langle 2, 1, 3\sqrt{3} \rangle$, $\bar{b} = \langle -1, 1, 2\sqrt{3} \rangle$, $\bar{c}= \langle -1, 1, 4\sqrt{3} \rangle$; the angle at $\bar{b}$
  \end{enumerate}

  \item Consider the triangle with vertices $\bar{a} = \bar{0}$, $\bar{b} = \langle 1, 2, 1 \rangle$, and $\bar{c} = \langle 0, \alpha, 1 \rangle$. Find the values of $\alpha$ such that the magnitude of the angle at $\bar{a}$ is:
  \begin{multicols}{2}
  \begin{enumerate}
    \item $\frac{\pi}{6}$
    \item $\frac{\pi}{4}$
    \item $\frac{\pi}{3}$
    \item $\arccos\left( \frac{2\sqrt{2}}{\sqrt{3}} \right)$
    \item $\frac{5\pi}{6}$
    \item $\frac{3\pi}{4}$
    \item $\frac{\pi}{2}$
  \end{enumerate}
  \end{multicols}
  \end{enumerate}
\end{exercisebox}

\begin{exercisebox}
  \begin{enumerate}[resume=AngleEX]  

  \item Consider a triangle with vertices $\bar{a}, \bar{b}, \bar{c}$. Determine the following:
  \begin{enumerate}
    \item $\|\bar{a} - \bar{c}\|$ if $\|\bar{a} - \bar{b}\| = 2$, $\|\bar{c} - \bar{b}\| = 3$, and the angle at $\bar{b}$ has magnitude $\frac{\pi}{6}$.
    \item $\|\bar{a} - \bar{b}\|$ if $\|\bar{a} - \bar{c}\| = 2$, $\|\bar{c} - \bar{b}\| = 3$, and the angle at $\bar{b}$ has magnitude $\frac{\pi}{3}$.
    \item $\|\bar{a} - \bar{b}\|$ if $\|\bar{a} - \bar{c}\| = 3$, $\|\bar{c} - \bar{b}\| = 2$, and the angle at $\bar{b}$ has magnitude $\frac{\pi}{2}$.
    \item $\|\bar{a} - \bar{b}\|$ if $\|\bar{a} - \bar{c}\| = 3$, the angle at $\bar{a}$ is $\frac{\pi}{3}$ and the angle at $\bar{c}$ is $\frac{\pi}{2}$.
  \end{enumerate}

  \item Let $\bar{a}, \bar{b}, \bar{c}$ be the vertices of a triangle. Show that if the angle at $\bar{a}$ has magnitude $\frac{\pi}{2}$, then $\|\bar{b} - \bar{c}\|^2 = \|\bar{a} - \bar{c}\|^2 + \|\bar{a} - \bar{b}\|^2$.

  \item Let $\bar{a}, \bar{b}, \bar{c}$ be the vertices of a triangle. Show that if $\|\bar{b} - \bar{c}\|^2 = \|\bar{a} - \bar{c}\|^2 + \|\bar{a} - \bar{b}\|^2$, then the angle at $\bar{a}$ has magnitude $\frac{\pi}{2}$.

  \item Let $\bar{a}, \bar{b}, \bar{c}$ be the vertices of a triangle. If the angles at $\bar{a}$ and $\bar{c}$ have the same magnitude, show that $\|\bar{b} - \bar{a}\| = \|\bar{b} - \bar{c}\|$.

  \item Let $\bar{a}, \bar{b}, \bar{c}$ be the vertices of a triangle. If $\|\bar{b} - \bar{a}\| = \|\bar{b} - \bar{c}\|$, show that the angles at $\bar{a}$ and $\bar{c}$ have the same magnitude.

  \item Let $\bar{a}, \bar{b}, \bar{c}$ be the vertices of a triangle. If $\|\bar{b} - \bar{a}\| = \|\bar{b} - \bar{c}\| = \|\bar{a} - \bar{c}\|$, show that the angles at $\bar{a}$, $\bar{b}$ and $\bar{c}$ all have the same magnitude.

  \item Let $\bar{a}, \bar{b}, \bar{c}$ be the vertices of a triangle. If the angles at $\bar{a}, \bar{b}, \bar{c}$ all have the same magnitude, show that $\|\bar{b} - \bar{a}\| = \|\bar{b} - \bar{c}\| = \|\bar{a} - \bar{c}\|$.
\end{enumerate}
\end{exercisebox}
